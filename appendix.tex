
\section{Supplementary material}


\subsection{Proof of Proposition~\ref{prop:PFA:undecidabilty}.\ref{prop:PFA:undecidabilty:low}}\label{proof:prop:PFA:undecidabilty:low}

A PFA is \emph{complete} if for all $s\in\S$ and $a\in\Act$, exists
some $\mu\in\Dist(\S)$ such that $s\reach{a}\mu$.
%
Notice that in a complete PFA, for $\plan\in\Act^*$ and
$\rho\in\execf$,
\begin{align}
  &\strat_\plan(\rho)(a,\mu)=1 \text{ \ iff \ } \bar{\rho}a\leq\plan \text{ \ (for a unique } \mu \text{), and} \label{eq:strat:plan:complete:i}\\
  &\strat_\plan(\rho)(\complete)=1 \text{ \ iff \ } \bar{\rho}=\plan. \label{eq:strat:plan:complete:ii}
\end{align}
%
As a consequence of this, we have the next lemma.

\begin{lemma}\label{lm:strat:plan:complete}
  In a complete PFA,
  $\Prob_{\sini}^{\strat_\plan}(\{{\rho\complete}\mid{\bar{\rho}=\plan}\})=1$,
  for all $\plan\in\Act^*$.
\end{lemma}
%
\begin{proof}
  First notice that $\Prob^{\strat_\plan}_s(\cexecf)=1$ (see
  \Cref{def:plan:compat} and following paragraphs).
  
  Suppose the lemma is false.  Then, there are $\plan\in\Act^*$ and
  $\rho\in\execf$ such that $\bar{\rho}\neq\plan$ and
  $\Prob_{\sini}^{\strat_\plan}(\{\rho\complete\})>0$.

  Suppose $\bar{\rho}\leq\plan$. Because of \cref{eq:def:prob:ii},
  $\strat_\plan(\rho)(\complete)> 0$, but then $\bar{\rho}=\plan$
  by \cref{eq:strat:plan:complete:ii} yielding a contradiction.

  Suppose $\plan\leq\bar{\rho}$. Hence,
  $\overline{\rho[..\size{\plan}]}=\plan$.  Because of
  \cref{eq:def:prob:ii}, $\strat_\plan(\rho[..\size{\plan}])(a,\mu)>0$
  for some $a\in\Act^*$ and $\mu\in\Dist(\S)$, but this contradicts
  \cref{eq:strat:plan:complete:ii}.

  Then, it necessarily exists $k\geq1$ such that
  $\bar{\rho}[k]\neq\plan[k]$.  By \cref{eq:def:prob:ii},
  $\strat_\plan(\rho[..\size{\plan}])(\bar{\rho}[k],\mu)>0$ for some
  $\mu\in\Dist(\S)$ which contradicts \cref{eq:strat:plan:complete:i},
  proving thus the lemma.
\end{proof}

Let $\pfa=\tup{\S,\Act,\ra,\V}$ be a complete PFA.  Let $\pfa^\star$
be exactly the same as $\pfa$ where the accepting states are exactly
those that are not accepting in $\pfa$, that is
$\pfa^\star=\tup{\S,\Act,\ra,\V^\star}$ with
$\V^\star(s)=(\V(s)\setminus\{\fin\})\cup\{\fin\mid{\fin\notin\V(s)}\}$.
%
Then, the set of accepting states of $\pfa^\star$ is
$F^\star=\S\setminus F$.

From \Cref{lm:strat:plan:complete}, it follows that for all $\plan\in\Act^*$,
%
\begin{align*}
  \Prob_{\pfa,\sini}^{\strat_\plan}(\Succ(\plan,F)) \hspace{-4em} & \\
  &  =
  1 - \Prob_{\pfa,\sini}^{\strat_\plan}(\{{\rho\complete}\mid{\bar{\rho}=\plan \text{ and } \last(\rho)\notin F}\}) \\
  & =
  1 - \Prob_{\pfa^\star,\sini}^{\strat_\plan}(\Succ(\plan,F^\star)).
\end{align*}
%
(We added a subscript in $\Prob$ to indicate in which PFA it
originates.)

As a consequence of this observation,
%$L(\pfa,q)\neq\emptyset$
there is a word $\plan\in\Act^*$ such that
$\Prob_{\pfa,\sini}^{\strat_\plan}(\Succ(\plan,F)) \geq q$
%
iff there is a word $\plan\in\Act^*$ such that
$\Prob_{\pfa^\star,\sini}^{\strat_\plan}(\Succ(\plan,F^\star)) < 1-q$.
%
Considering
\Cref{prop:PFA:undecidabilty}.\ref{prop:PFA:undecidabilty:high}, this
yields the following proposition
(i.e. \Cref{prop:PFA:undecidabilty}.\ref{prop:PFA:undecidabilty:low}).

\begin{proposition}
  The following problem is undecidable:
  %
  Given a PFA and an upper bound $q\in[0,1]$, determine if there is a
  word $\plan\in\Act^*$ such that
  $\Prob_{\sini}^{\strat_\plan}(\Succ(\plan,F)) < q$.
\end{proposition}


\subsection{Proof of \Cref{lm:mc:core} (\cref{lm:mc:core:ii})}

For \cref{lm:mc:core:ii}, let $\strat$ be a strategy for
$\model$ and define strategy $\strat^\star$ for
$\model{\times}\dfa$ as follows.
%
Let
$\rho=(s_0,t_0)\, a_1\, (s_1,t_1)\ldots (s_{n-1},t_{n-1})\, a_n\, (s_n,t_n)\in\setRini$
and $0\leq k\leq\size{\rho}$, then:
%
\begin{enumerate}
\item%
  If $k=\size{\rho}$, let $\strat^\star(\rho)(\complete)=1$.
\item%
  If instead $k<\size{\rho}$, let
  $\strat^\star(\rho[..k])(a_{k+1},\mu\otimes\Dirac_{t_{k+1}})=\strat(\proj(\rho[..k]))(a_{k+1},\mu)$
  for all $\mu\in\Dist(\S)$.
\end{enumerate}
%
Since $\proj$ is a bijection from $\setRini$ to $\setSini$, 
$\strat^\star$ is well defined for all executions that are prefixes of
some execution in $\setRini$.
%
For any other execution, $\strat^\star$ can be defined in any way.

Following \cref{eq:def:prob:i,eq:def:prob:ii}, it should not be hard
to see that
%
$\Prob^{\strat}_{\sini}(\Cyl(\proj(\rho))) =
\Prob^{\strat^\star}_{(\sini,\tini)}(\{\rho\complete\}) =
\Prob^{\strat^\star}_{\sini}(\Cyl(\rho))$
%
Hence, from \cref{eq:prob:R:bigcup:cyl,eq:prob:S:bigcup:cyl},
\cref{lm:mc:core:ii} follows.
