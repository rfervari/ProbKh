
\section*{Supplementary material: Complete proofs}\stepcounter{section}

\subsection{Proof of Proposition~\ref{prop:PFA:undecidabilty} (\cref{prop:PFA:undecidabilty:low})}\label{proof:prop:PFA:undecidabilty:low}

A PFA is \emph{complete} if for all $s\in\S$ and $a\in\Act$, exists
some $\mu\in\Dist(\S)$ such that $s\reach{a}\mu$.
%
Notice that in a complete PFA, for $\plan\in\Act^*$ and
$\rho\in\execf$,
\begin{align}
  &\strat_\plan(\rho)(a,\mu)=1 \text{ \ iff \ } \bar{\rho}a\leq\plan \text{ \ (for a unique } \mu \text{), and} \label{eq:strat:plan:complete:i}\\
  &\strat_\plan(\rho)(\complete)=1 \text{ \ iff \ } \bar{\rho}=\plan. \label{eq:strat:plan:complete:ii}
\end{align}
%
As a consequence of this, we have the next lemma.

\begin{lemma}\label{lm:strat:plan:complete}
  In a complete PFA,
  $\Prob_{\sini}^{\strat_\plan}(\{{\rho\complete}\mid{\bar{\rho}=\plan}\})=1$,
  for all $\plan\in\Act^*$.
\end{lemma}
%
\begin{proof}
  First notice that $\Prob^{\strat_\plan}_s(\cexecf)=1$ (see
  \Cref{def:plan:compat} and following paragraphs).
  
  Suppose the lemma is false.  Then, there are $\plan\in\Act^*$ and
  $\rho\in\execf$ such that $\bar{\rho}\neq\plan$ and
  $\Prob_{\sini}^{\strat_\plan}(\{\rho\complete\})>0$.

  Suppose $\bar{\rho}\leq\plan$. Because of \cref{eq:def:prob:ii},
  $\strat_\plan(\rho)(\complete)> 0$, but then $\bar{\rho}=\plan$
  by \cref{eq:strat:plan:complete:ii} yielding a contradiction.

  Suppose $\plan\leq\bar{\rho}$. Hence,
  $\overline{\rho[..\size{\plan}]}=\plan$.  Because of
  \cref{eq:def:prob:ii}, $\strat_\plan(\rho[..\size{\plan}])(a,\mu)>0$
  for some $a\in\Act^*$ and $\mu\in\Dist(\S)$, but this contradicts
  \cref{eq:strat:plan:complete:ii}.

  Then, it necessarily exists $k\geq1$ such that
  $\bar{\rho}[k]\neq\plan[k]$.  By \cref{eq:def:prob:ii},
  $\strat_\plan(\rho[..\size{\plan}])(\bar{\rho}[k],\mu)>0$ for some
  $\mu\in\Dist(\S)$ which contradicts \cref{eq:strat:plan:complete:i},
  proving thus the lemma.
\end{proof}

Let $\pfa=\tup{\S,\Act,\ra,\V}$ be a complete PFA.  Let $\pfa^\star$
be exactly the same as $\pfa$ where the accepting states are exactly
those that are not accepting in $\pfa$, that is
$\pfa^\star=\tup{\S,\Act,\ra,\V^\star}$ with
$\V^\star(s)=(\V(s)\setminus\{\fin\})\cup\{\fin\mid{\fin\notin\V(s)}\}$.
%
Then, the set of accepting states of $\pfa^\star$ is
$F^\star=\S\setminus F$.

From \Cref{lm:strat:plan:complete}, it follows that for all $\plan\in\Act^*$,
%
\begin{align*}
  \Prob_{\pfa,\sini}^{\strat_\plan}(\Succ(\plan,F)) 
  &  =
  1 - \Prob_{\pfa,\sini}^{\strat_\plan}(\{{\rho\complete}\mid{\bar{\rho}=\plan \text{ and } \last(\rho)\notin F}\}) \\
  & =
  1 - \Prob_{\pfa^\star,\sini}^{\strat_\plan}(\Succ(\plan,F^\star)).
\end{align*}
%% \begin{align*}
%%   \Prob_{\pfa,\sini}^{\strat_\plan}(\Succ(\plan,F)) \hspace{-4em} & \\
%%   &  =
%%   1 - \Prob_{\pfa,\sini}^{\strat_\plan}(\{{\rho\complete}\mid{\bar{\rho}=\plan \text{ and } \last(\rho)\notin F}\}) \\
%%   & =
%%   1 - \Prob_{\pfa^\star,\sini}^{\strat_\plan}(\Succ(\plan,F^\star)).
%% \end{align*}
%
(We added a subscript in $\Prob$ to indicate in which PFA it
originates.)

As a consequence of this observation,
%$L(\pfa,q)\neq\emptyset$
there is a word $\plan\in\Act^*$ such that
$\Prob_{\pfa,\sini}^{\strat_\plan}(\Succ(\plan,F)) \geq q$
%
iff there is a word $\plan\in\Act^*$ such that
$\Prob_{\pfa^\star,\sini}^{\strat_\plan}(\Succ(\plan,F^\star)) < 1-q$.
%
Considering
\Cref{prop:PFA:undecidabilty}.\ref{prop:PFA:undecidabilty:high}, this
yields the following proposition
(i.e. \Cref{prop:PFA:undecidabilty}.\ref{prop:PFA:undecidabilty:low}).

\begin{proposition}
  The following problem is undecidable:
  %
  Given a PFA and an upper bound $q\in[0,1]$, determine if there is a
  word $\plan\in\Act^*$ such that
  $\Prob_{\sini}^{\strat_\plan}(\Succ(\plan,F)) < q$.
\end{proposition}



\subsection{Proof of Proposition~\ref{prop:nonprob:prob}}\label{proof:prop:nonprob:prob}

\begin{proof}
  Observe that
  %
  $\Succ(\plan,S) = \{{\rho\complete}\mid{\bar{\rho}=\plan}\}$
  %
  and hence
  %
  \begin{equation}\label{eq:proof:nonprob:prob:probsucc}
    \textstyle
    \Prob^\strat_s(\Succ(\plan,\S)) =
    \Prob^\strat_s(\{{\rho\complete}\mid{\bar{\rho}=\plan}\wedge{\first(\rho)=s}\})
    %\sum_{\substack{\bar{\rho}=\plan\\\first(\rho)=s}}\Prob^\strat_s(\{\rho\complete\})
  \end{equation}
  %
%%   \[\Prob^\strat_s(\Succ(\plan,\S)) =
%%   \Prob^\strat_s(\{{\rho\complete}\mid{\bar{\rho}=\plan}\}) =
%%   \sum_{\substack{\bar{\rho}=\plan\\\first(\rho)=s}}\Prob^\strat_s(\{\rho\complete\})\]
%%   %
  for all $s\in\S$ and $\strat\in\Comp(\plan)$. In particular, the
  equality considers the fact that
  $\Prob^\strat_s(\{\rho\complete\})=0$ if $s\neq\first(\rho)$.

  We first show implication $(\Rightarrow)$ of
  \cref{prop:nonprob:prob:i}.
  %
  For this suppose $\plan$ is SE at $s$ and
  $\inf_{\strat\in\Comp(\plan)}\Prob^\strat_s(\Succ(\plan,\S))<1$.
  %
  Because of \cref{eq:proof:nonprob:prob:probsucc}, there must exists
  $\strat\in\Comp(\plan)$ and $\rho\in\execf$ such that
  $\bar{\rho}\neq\pi$, $\first(\rho)=s$, and
  $\Prob^\strat_s(\{\rho\complete\})>0$~$(\star)$.
  %
  Say $\rho=s_0\, a_1\, s_1\ldots s_{n-1}\, a_n\, s_n$ with $s_0=s$.

  Suppose first that $\bar{\rho}\leq\plan$ (the prefix should be
  proper).  Because of \cref{eq:def:prob:ii}, necessarily
  $\strat(\rho)(\complete) > 0$.  Besides, since $\plan$ is SE for $s$
  and $s\reach{\bar{\rho}}s_n$ then, for some $t\in\S$,
  $s_n\reach{\plan[n{+}1]}t$ (i.e.\ $s_n\reach{\plan[n{+}1]}\Dirac_t$).
  %
  Because $\bar{\rho}(\plan[n{+}1])\in\pref(\plan)$,
  $\bar{\rho}\{a\mid{\last(\rho)\reach{a}\mu}\}\cap\pref(\plan)\neq\emptyset$.
  Therefore, by \Cref{def:plan:compat}, $\strat(\rho)(\complete) = 0$
  inducing a contradiction.

  So $\bar{\rho}\notin\pref(\plan)$ and either $\plan\leq\bar{\rho}$
  (properly) or exists some $1\leq i \leq\min(\size{\plan},\size{\bar{\rho}})$
  such that $\plan[i]\neq\bar{\rho}[i]$.

  If $\plan\leq\bar{\rho}$, then $\bar{\rho}[..\size{\plan}]=\plan$ and
  hence, by \Cref{def:plan:compat}, necessarily
  $\strat(\rho[..\size{\plan}])(\complete) = 1$.  By \cref{eq:def:prob:ii},
  this implies that $\Prob^\strat_s(\{\rho\complete\})=0$,
  contradicting $(\star)$.

  If instead $\plan[i]\neq\bar{\rho}[i]=a_i$ with
  $1\leq i \leq\min(\size{\plan},\size{\bar{\rho}})$, by \cref{eq:def:prob:ii},
  $\strat(\rho[..(i-1)])(a_i,\mu) > 0$ for some
  $\mu\in\Dist(\S)$ with $s_i\reach{a_i}\mu$.
  However $\overline{\rho[..i]}a_i \notin \pref(\plan)$
  (because $\plan[i]\neq a_i$) contradicting the assumption that
  $\strat\in\Comp(\plan)$ (by \Cref{def:plan:compat}).

  % This proves implication $(\Rightarrow)$ of \cref{prop:nonprob:prob:i}.

  For implication $(\Leftarrow)$ of \cref{prop:nonprob:prob:i} suppose
  $s\reach{\plan[..i]}t$ for $i<\size{\plan}$, that is, there are $s_0$,
  $s_1$, \ldots $s_i$ such that $s=s_0$, $t=s_i$, and
  $s_k\reach{\plan[k{+}1]}\Dirac_{s_{k{+}1}}$ for all $0\leq k < i$.
  %
  Take a $\plan$-compatible strategy $\strat$ such that, for all
  $0\leq k < i$,
  $\strat(s_0\,\plan[1]\,s_1\ldots s_{k{-}1}\,\plan[k]\,s_k)(\plan[k{+}1],\Dirac_{s_{k{+}1}})=1$.
  Notice that such strategy exists and hence
  $\Prob^\strat_s(\Succ(\plan,\S))=1$ by assumption.
  %
  By \cref{eq:proof:nonprob:prob:probsucc} and \cref{eq:def:prob:ii}
  there must exist some $\rho\in\execf$ such that $\bar{\rho}=\plan$,
  $s_0\,\plan[1]\,s_1\ldots s_{k{-}1}\,\plan[i]\,s_i = \rho[..i]$ and
  $\Prob^\strat_s(\{\rho\complete\})>0$.  By \cref{eq:def:prob:ii}, in
  particular, there exists $s_{i{+}1}$ such that
  $\strat(\rho[..i])(\plan[i{+}1],\Dirac_{s_{i{+}1}})> 0$ which
  implies that $s_i\reach{\plan[i{+}1]}\Dirac_{s_{i{+}1}}$, or
  equivalently, $t\reach{\plan[i{+}1]}s_{i{+}1}$, since $t=s_i$.
  %
  Therefore $\plan$ is SE in $s$, which proves~\cref{prop:nonprob:prob:i}.

  For \cref{prop:nonprob:prob:ii}, it suffices to show that for all
  $s\in A$,
  \[
    \plan \text{ is SE at } s \text{ and } s \reach{\plan} t \text{ implies }
    t\in G \quad
    \text{iff}\quad
    \inf_{\strat\in\Comp(\plan)}\Prob^\strat_s(\Succ(\plan,G))=1.
  \]
%%   \begin{quote}
%%     $\plan$ is SE at $s$ and $s \reach{\plan} t$ implies
%%     $t\in G$ \newline
%%     \mbox{}\hfill
%%     iff\quad
%%     $\inf_{\strat\in\Comp(\plan)}\Prob^\strat_s(\Succ(\plan,G))=1$.
%%   \end{quote}
  
  For the implication $(\Rightarrow)$, assume $\plan$ is SE at $s$ and
  $s \reach{\plan} t$ implies $t\in G$.
  %
  Suppose
  $\inf_{\strat\in\Comp(\plan)}\Prob^\strat_s(\Succ(\plan,G))<1$.
  Since
  $\inf_{\strat\in\Comp(\plan)}\Prob^\strat_s(\Succ(\plan,\S))=1$ (by
  \cref{prop:nonprob:prob:i}), there must be a strategy
  $\strat\in\Comp(\plan)$ such that
  $\Prob^\strat_s(\{\rho\complete\})>0$, for some $\rho\in\execf$ with
  $\bar{\rho}=\plan$ and $\first(\rho)=s$ (because of
  \cref{eq:proof:nonprob:prob:probsucc}) and also $\last(\rho)\notin G$.
  %
  By \cref{eq:def:prob:i,eq:def:prob:ii}, also
  $\Prob^\strat_s(\{\rho\})>0$ from which $s\reach{\plan}\last(\rho)$,
  contradicting the assumption that $s \reach{\plan} t$ implies
  $t\in G$.

  For $(\Leftarrow)$, suppose
  $\inf_{\strat\in\Comp(\plan)}\Prob^\strat_s(\Succ(\plan,G))=1$.
  Since $G\subseteq\S$,
  $\inf_{\strat\in\Comp(\plan)}\Prob^\strat_s(\Succ(\plan,\S))=1$, and
  hence $\plan$ is SE at $s$ by \cref{prop:nonprob:prob:i}.
  %
  Suppose now that $s \reach{\plan} t$, that is, there are $s_0$,
  $s_1$, \ldots $s_{\size{\plan}}$ such that $s=s_0$, $t=s_{\size{\plan}}$, and
  $s_k\reach{\plan[k{+}1]}\Dirac_{s_{k{+}1}}$ for all $0\leq k < \size{\plan}$.
  %
  We can construct a $\plan$-compatible strategy $\strat$
  such that, for all $0\leq k < \size{\plan}$,
  $\strat(s_0\,\plan[1]\,s_1\ldots s_{k{-}1}\,\plan[k]\,s_k)(\plan[k{+}1],\Dirac_{s_{k{+}1}})=1$,
  and, in addition, for
  $\rho = s_0\,\plan[1]\,s_1\ldots s_{\size{\plan}{-}1}\,\plan[k]\,s_{\size{\plan}}$,
  $\strat(\rho)(\complete)=1$.
  %
  Thus $\Prob^\strat_s(\{\rho\bot\})>0$.  Since
  $\Prob^\strat_s(\Succ(\plan,G))=1$, $\rho\bot\in\Succ(\plan,G)$
  and therefore $t=\last(\rho)\in G$, which finally proves
  \cref{prop:nonprob:prob:ii}.
\end{proof}




\subsection{Proof of \Cref{lm:plan:plans}}


\begin{proof}
  For \cref{lm:plan:plans:i}, suppose by contradiction that
  $\inf_{\strat\in\Comp(\plans)}\Prob^\strat_s(\Succ(\plans,G))<1$.
  %
  Then exist $\strat\in\Comp(\plans)$ and $\rho\in\execf$ with
  $\Prob^\strat_s(\{\rho\complete\})>0$ and either
  $\bar{\rho}\notin\plans$ or $\last(\rho)\notin G$.
  %
  Say $\rho=s_0\, a_1\, s_1\ldots s_{n-1}\, a_n\, s_n$.  Necessarily
  $s_0=s$.
  %
  Define the strategy $\strat^\star$ such that for all $0\leq k<n$,
  $\strat^\star(\rho[..k])(a_{i+1},\Dirac_{s_{i+1}})=1$ and
  $\strat^\star(\rho)(\complete)=1$.  Notice that
  $\Prob^{\strat^\star}_s(\{\rho\complete\})=1$

  Suppose $\bar{\rho}=\plan$ for some $\plan\in\plans$ and
  $\last(\rho)\notin G$.  Then $\strat^\star$ is $\plan$-compatible
  and $\Prob^{\strat^\star}_s(\Succ(\plan,G))=0$ contradicting the
  hypothesis.

  So, inevitably, $\bar{\rho}\notin\plans$.
  %
  Suppose $\bar{\rho}\notin\pref(\plans)$.  Hence, for all
  $\plan\in\plans$ exists some $0<k\leq n$ such that
  $\bar{\rho}[..k{-}1]=\plan[..k{-}1]$ and $a_k\neq\plan[k]$.  Let
  $\hat{k}$ be the largest of such $k$'s.
  %
  Then $\bar{\rho}[..k{-}1]\in\pref(\plans)$ and
  $\strat(\rho[..k{-}1])(a_k,\Dirac_{s_k})>0$.  However
  $\bar{\rho}[..k{-}1]a_k\notin\pref(\plans)$ contradicting the
  assumption that $\strat$ is $\plans$-compatible.

  Therefore, there is some $\plan\in\plans$ such that
  $\bar{\rho}\leq\plan$ but $\bar{\rho}\notin\plans$.
  %
  Because $\Prob^\strat_s(\{\rho\complete\})>0$,
  $\strat(\rho)(\complete)>0$ and, since $\strat$ is
  $\plans$-compatible, there is no $a\in\Act$ such that
  $\bar{\rho}a\in\pref(\plans)$ and $\last(\rho)\reach{a}\mu$ for some
  $\mu\in\Dist(\S)$.
  %
  In particular $\bar{\rho}a\neq\plan$ for all $a\in\Act$.  Then
  $\strat^\star$ is $\plan$-compatible and
  $\Prob^{\strat^\star}_s(\Succ(\plan,G))=0$ contradicting the
  hypothesis, proving thus \cref{lm:plan:plans:i}.

  \Cref{lm:plan:plans:ii} follows directly from \cref{lm:plan:plans:i}.
\end{proof}




\subsection{Proof of \Cref{lm:mc:core} (\cref{lm:mc:core:ii})}

\begin{proof}
  For \cref{lm:mc:core:ii}, let $\strat$ be a strategy for
  $\model$ and define strategy $\strat^\star$ for
  $\model{\times}\dfa$ as follows.
  %
  Let
  $\rho=(s_0,t_0)\, a_1\, (s_1,t_1)\ldots (s_{n-1},t_{n-1})\, a_n\, (s_n,t_n)\in\setRini$
  and $0\leq k\leq\size{\rho}$, then:
  %
  \begin{enumerate}
  \item%
    If $k=\size{\rho}$, let $\strat^\star(\rho)(\complete)=1$.
  \item%
    If instead $k<\size{\rho}$, let
    $\strat^\star(\rho[..k])(a_{k+1},\mu\otimes\Dirac_{t_{k+1}})=\strat(\proj(\rho[..k]))(a_{k+1},\mu)$
    for all $\mu\in\Dist(\S)$.
  \end{enumerate}
  %
  Since $\proj$ is a bijection from $\setRini$ to $\setSini$, 
  $\strat^\star$ is well defined for all executions that are prefixes of
  some execution in $\setRini$.
  %
  For any other execution, $\strat^\star$ can be defined in any way.

  Following \cref{eq:def:prob:i,eq:def:prob:ii}, it should not be hard
  to see that
  %
  $\Prob^{\strat}_{\sini}(\Cyl(\proj(\rho))) =
  \Prob^{\strat^\star}_{(\sini,\tini)}(\{\rho\complete\}) =
  \Prob^{\strat^\star}_{\sini}(\Cyl(\rho))$
  %
  Hence, from \cref{eq:prob:R:bigcup:cyl,eq:prob:S:bigcup:cyl},
  \cref{lm:mc:core:ii} follows.
\end{proof}
