\section{Introduction}
\label{sec:intro}

Here we list some important pieces of work, motivating ours:

\begin{itemize}
    \item Knowing how has been investigated is the last years, from different perspectives, especially by combining epistemic operators of knowing that with operators describing abilities~\cite{Mccarthy69,Moore85,Les00,Hoek00,HerzigT06}. This is not considered as a proper reading~\cite{JamrogaA07,Herzig15}
    \item In~\cite{Wang15lori,Wang16,Wang2016} a new perspective on knowing how emerged, in which a new modality is specifically defined with the purpose of capturing this concept.
    \item This raised a family of logics, witnessed by all related work (see e.g.~\cite{LiWang17,Li17,Li17bis,FervariHLW17,LiW21,NaumovT17,NaumovT18,NaumovT19,Naumov2018a}).
    \item A notion of `epistemic indistinguishability' is missing, arguably fixed by~\cite{AFSVQ21,AFSVQ23}.
    \item With this at hand, it was possible to define dynamic epistemic modalities (e.g.~\cite{AFSV22}).
    \item Constraints on plans, like regularity or budget constraints~\cite{DemriF23}.
    \item The latter opens the path to study other constraints, in particular, \emph{knowing how to achieve a goal with a certain probability}.
    \item Relate to other epistemic based logics with probabilities, and with the version of knowing how with uncertainty~\cite{NaumovT19}. Recall the differences, and the case of use that we are able to capture.
    \item Relate to planning with probabilities.
    \item Model-checking with probabilities \cite{BA95,TJ07,BaierAFK18}
    \item Recall the different versions of our modality, how we obtain a decidable logic, and why it makes sense.
    \item Connections with reinforcement learning and reasoning about such scenarios.
\end{itemize}