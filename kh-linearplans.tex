\section{Knowing How with Linear Plans}
\label{sec:khlinearplans}

We start by introducing the most basic notion of knowing how as defined in e.g.~\cite{Wang15lori,Wang16,Wang2016}.


\begin{definition}\label{def:lts}
    Let $\PROP$ be a countable set of propositional symbols. 
    A \emph{Labeled Transition System (LTS)}  is a tuple
    $\model=\tup{\S,\ACT,\ra,\V}$ such that:
    \begin{itemize}
        \item $\S$ is a countably non-empty set of states,
        \item $\ACT$ is a countable set of action symbols,
        \item ${\ra} \subseteq \S \times \ACT \times \dist(\S)$ is a transition relation,
    \end{itemize}
    Elements of $\ACT^*$ are called \emph{plans}.
\end{definition}

\begin{definition}
    \label{def:syntax}
    The set of formulas (a.k.a. the language) of $\Khlogic$ is defined by the following BNF:
    \[
        \varphi, \psi ::= p \mid \neg \varphi \mid \varphi \vee \psi \mid \kh(\psi,\varphi),
    \]
    where $p\in\PROP$. Other Boolean operators are defined as usual. Formulas of the form $\kh_i(\psi,\varphi)$ are read as \emph{``agent $i$ knows how to achieve $\varphi$ given $\psi$''}
\end{definition}

\begin{definition} \label{def:semantics-kh}
    Let $\model = \tup{\S,\ACT,\ra,\V}$ be an LTS and let $s\in\S$, the satisfiability relation $\models$ for $\Khlogic$ is inductively defined as:
    \[
    \begin{array}{l@{\ \ \ }c@{\ \ \  }l}
    \model, s \models p & \iffdef & p \in \V(s) \\
    \model, s\models \neg\varphi & \iffdef & \model, s \not\models \varphi \\
    \model, s \models \psi\vee\varphi & \iffdef & \model, s \models \psi \mbox{ or }\model, w \models \varphi \\
    \model, s \models \kh(\psi,\varphi) & \iffdef & \text{there is } \plan \in \ACT^* \;\text{such that:} \\
    & & \ \ \text{\rm (1)} \ \plan \text{ is SE-executable at }  \truthset{\model}{\psi}\; \text{and} \\
    & & \ \ \text{\em (2)} \ \truthset{\model}{\psi} \reach{\plan} \truthset{\model}{\varphi}, 
    \end{array}
    \]      where: $\truthset{\model}{\chi} := \csetsc{s\in\S}{\model,w\models\chi}$. Define: $\model\models\varphi$ iff  $\truthset{\model}{\varphi}=\S$, and $\models\varphi$ iff $\model\models\varphi$, for all LTS $\model$.
\end{definition}

\begin{definition}\label{def:plts}
    Let $\PROP$ be a countable set of propositional symbols. 
    A \emph{Probabilistic Labeled Transition System (PLTS)}  is a tuple
    $\model=\tup{\S,\ACT,\dist(\S),\ra,\V}$ such that $\tup{\S,\ACT,\ra,\V}$ is an LTS and $\dist(\S)$ is the set of probability distributions over $\S$.
\end{definition}