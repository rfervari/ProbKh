\section{Knowing How with Linear Plans}
\label{sec:khlinearplans}

\subsection{Basic Definitions}

We start by introducing the most basic notion of knowing how as defined in e.g.~\cite{Wang15lori,Wang16,Wang2016}. Formulas describing the abilities of an agent of achieving a certain goal, are interpreted over Labeled Transition Systems, which indicate what actions are available for execution at each state, and how they transform one state into another.

In order to determine when an agent knows how to achieve a goal, we need to characterize those sequences of actions (or plans) that result appropriate for such a purpose. This is the notion of \emph{strongly executable} plans about, indicating that a plan is ``fail proof''. This notion, discussed already in~\cite{Wang15lori,Wang16,Wang2016} was inspired by conformant planning (see e.g.~\cite{Smith&Weld98,Bonet2010}).

\begin{definition}\label{def:plans}
    Let $\model=\tup{\S,\Act,\ra,\V}$ be an LTS. 
    Elements of $\Act^*$ are called \emph{plans} (with $\epsilon$ the empty plan).  Let $\plan\in\Act^*$, $\size{\plan}$ denotes its length ($\size{\epsilon}:=0$).
    For  $0\leq i \leq \size{\plan}$, the plan $\plan_i$ denotes the initial segment of $\plan$ up to (and including) the $i^{th}$ position (with $\plan_0 := \epsilon$). The action $\plan[i]$ is the one appearing in $\plan$ at the $i^{th}$ position. We define $\reach{\plan}$ as the composition $\reach{\plan[1]} \comp \ldots \comp \reach{\plan[\size{\plan}]}$. 

    We say that a plan $\plan\in\Act^*$ is \emph{strongly executable (SE)} at a state $s\in\S$ if and only if, for all $0\leq i \leq \size{\plan}-1$ and all $t\in\S$ such that $s\reach{\plan_i} t$, there is $v\in\S$ such that $t\reach{\plan[i+1]} v$. The plan $\plan$ is SE at $A\subseteq \S$ if and only if it is SE at every $s\in A$. The notation $A \reach{\plan} G$ (for $A,G\subseteq\S$) indicates that for all $s\in A$,  $s\reach{\plan} t$ implies $t\in G$.
\end{definition}

Now we are ready to introduce the language of knowing how.

\begin{definition}
    \label{def:syntax}
    The set of formulas (a.k.a. the language) of $\Khlogic$ is defined by the following BNF:
    \[
        \varphi, \psi ::= p \mid \neg \varphi \mid \varphi \vee \psi \mid \kh(\psi,\varphi),
    \]
    where $p\in\Prop$. Other Boolean operators are defined as usual. Formulas of the form $\kh(\psi,\varphi)$ are read as \emph{``the agent knows how to achieve $\varphi$ given $\psi$''}
\end{definition}

Formulas are interpreted over pointed LTS, i.e., w.r.t. an LTS and a given state. 

\begin{definition} \label{def:semantics-kh}
    Let $\model = \tup{\S,\Act,\ra,\V}$ be an LTS and let $s\in\S$, the satisfiability relation $\models$ for $\Khlogic$ is inductively defined as:
    \[
    \begin{array}{l@{\ \ \ }c@{\ \ \  }l}
    \model, s \models p & \iffdef & p \in \V(s) \\
    \model, s\models \neg\varphi & \iffdef & \model, s \not\models \varphi \\
    \model, s \models \psi\vee\varphi & \iffdef & \model, s \models \psi \mbox{ or }\model, w \models \varphi \\
    \model, s \models \kh(\psi,\varphi) & \iffdef & \text{there is } \plan \in \Act^* \;\text{such that:} \\
    & & \ \ \text{\rm (1)} \ \plan \text{ is SE at }  \truthset{\model}{\psi}\; \text{and} \\
    & & \ \ \text{\em (2)} \ \truthset{\model}{\psi} \reach{\plan} \truthset{\model}{\varphi}, 
    \end{array}
    \]      where: $\truthset{\model}{\chi} := \csetsc{s\in\S}{\model,w\models\chi}$. Define: $\model\models\varphi$ iff  $\truthset{\model}{\varphi}=\S$, and $\models\varphi$ iff $\model\models\varphi$, for all LTS $\model$.
\end{definition}

The model-checking problem for a given logic is defined as follows, where models and formulas are instantiated with those corrresponding to each particular case. 

\begin{description} \itemsep 0cm
    \item[Input:] A model $\model$, a state $s$ in $\model$ and a formula $\varphi$;
    \item[Output:] $\model,s\models\varphi$?
\end{description}

\begin{proposition}[\cite{DemriF23}]
    The model-checking problem for $\Khlogic$ is \PSPACE-complete.
\end{proposition}

\subsection{Indistinguishability Classes in Knowing How}

\begin{definition}[Uncertainty-based \lts]\label{def:ults}
    An \emph{uncertainty-based \lts} (LTSU) for $\Prop$, $\Act$ and $\AGT$ is a tuple     $\model=\tup{\S,\Act,\ra,\sim,\V}$ s.t. $\tup{\S,\Act,\ra,\V}$ is an LTS, and ${\sim}\subseteq \DS{}\times \DS{}$ (where $\DS{}\subseteq\Act^*$) is an equivalence relation over $\DS{}$ (called the indistinguishability relation between plans). 

    By $[\plan]_{\sim}:=\set{\plan' \in \DS{} \mid \plan \sim_i \plan'}$ we denote $\plan$'s equivalence relation with respect to $\DS{}$, then we define the \emph{indistinguishability set of $\model$} as $\Unc := \set{[\plan]_{\sim} \mid \plan\in\DS{}}$. 
        For simplicity sake, we sometimes denote $\model=\tup{\S,\Act,\ra,\Unc,\V}$ to refer to an LTSU, i.e., we will use its uncertainty set instead of the indistinguishability relation.
    \end{definition}
    
    Intuitively, $\DS{} = \bigcup_{\plans \in \Unc} \plans$ is the set of plans that the  agent is aware she has at her disposal, and each $\plans \in \Unc$ is an indistinguishability class. 
    
    Given her uncertainty over $\Act^*$, the abilities of the agent depend not on what a single plan can achieve, but rather on what a set of them can guarantee.
    
    \medskip
    
    \begin{definition}
       Let $\plans \subseteq \Act^*$ and $A,G \subseteq \S$, we write $A\reach{\plans} G$ if for all $s\in A$, $\plan\in\plans$ and $t\in \S$,  $s \reach{\plan} t$ implies $t\in G$. 
    \end{definition}
    
    \medskip
    
    In what follows, we introduce the notion of strong executability of plans (see, e.g.,~\cite{Wang15lori,AFSVQ23}), a condition which determines that a given plan or a set of them, is appropriate in order to achieve a certain goal.
    
    \medskip
    
    
    \begin{definition}[Strong executability of plans]\label{def:plans-exec}
    Let $\model=\tup{\S,\Act,\ra,\Unc,\V}$ be an LTSU. A \emph{set of plans} $\plans \subseteq \Act^*$ is \emph{strongly executable} at $u \in \S$ if and only if \emph{every} plan $\plan \in \plans$ is \emph{strongly executable} at~$u$.
    % $\stexec(\plans) = \bigcap_{\plan \in \plans} \stexec(\plan)$ is the set of the states in $\W$ where $\plans$ is SE.
    \end{definition}

    \begin{definition} \label{def:semantics-kh-uncertain}
        Let $\model = \tup{\S,\Act,\ra,\Unc,\V}$ be an LTSU and let $s\in\S$, the satisfiability relation $\models$ for $\Khlogic$ is inductively defined as:
        \[
        \begin{array}{l@{\ \ \ }c@{\ \ \  }l}
        \model, s \models \kh(\psi,\varphi) & \iffdef & \text{there is } \plans \in \Unc \;\text{such that:} \\
        & & \ \ \text{\rm (1)} \ \plans \text{ is SE at }  \truthset{\model}{\psi}\; \text{and} \\
        & & \ \ \text{\em (2)} \ \truthset{\model}{\psi} \reach{\plans} \truthset{\model}{\varphi}, 
        \end{array}
        \]      where: $\truthset{\model}{\chi} := \csetsc{s\in\S}{\model,w\models\chi}$. Define: $\model\models\varphi$ iff  $\truthset{\model}{\varphi}=\S$, and $\models\varphi$ iff $\model\models\varphi$, for all LTS $\model$.
    \end{definition}
    

    \begin{proposition}[\cite{AFSVQ21,AFSVQ23}]
        The model-checking problem for $\Khlogic$ over LTSU is in \PTIME.
    \end{proposition}
