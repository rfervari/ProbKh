\newtheorem{theorem}{Theorem}
\newtheorem{proposition}{Proposition}%[section]
\newtheorem{corollary}{Corollary}%[section]
\newtheorem{lemma}{Lemma}%[section]
\newtheorem{fact}{Fact}%[section]
\newtheorem{remark}{Remark}%[section]
\newtheorem{claim}{Claim}%[section]

\newtheorem{definition}{Definition}%[section]
\newtheorem{example}{Example}%[section]

%------------------------------------------------------------------------------------------------

\usepackage{cleveref}

% Abbreviated Cref
\Crefname{algorithm}{Alg.}{Algs.}
\Crefname{definition}{Def.}{Defs.}
% \Crefname{equation}{Eq.}{Eqs.}
% \Crefname{fact}{Fact}{Facts}
\Crefname{figure}{Fig.}{Figs.}
\Crefname{proposition}{Prop.}{Props.}
% \Crefname{lemma}{Lemma}{Lemmas}
\Crefname{theorem}{Thm.}{Thms.}
\Crefname{example}{Ex.}{Exs.}
\Crefname{corollary}{Cor.}{Cors.}
% \Crefname{enumi}{Item}{Items}
\Crefname{section}{Sec.}{Secs.}
\Crefname{appendix}{App.}{Apps.}
% \Crefname{table}{Table}{Tables}
% \Crefname{inlineenumi}{Item}{Items}
% % \Crefname{cond-khi}{}{}
% % \Crefname{inline-cond-khi}{}{}
% \Crefname{table}{Tab.}{Tabs.}



%% \usepackage{tikz}
%% \usetikzlibrary{arrows,decorations,shapes,automata,positioning,decorations.pathmorphing,patterns}

\usepackage{tikz, tikzscale, pgfplots}
\usetikzlibrary{arrows,arrows.meta,calc,automata,positioning,decorations.pathreplacing,shapes.geometric,shapes.misc,graphs,backgrounds,shadows.blur,snakes}

\tikzset{align at top/.style={baseline=(current bounding box.north)}}
\tikzstyle{every node}=[font=\scriptsize]

\tikzset{
  every picture/.style = {
    thick,
    ->,
    >=stealth',
    node distance = 1.5em and 3em,
  }
  ,
  cross line/.style = {
    preaction = {
      draw=white,
      -,
      line width=4pt
    }
  }
  , % states
  state/.style = {
    circle,
%    rectangle,
%    rounded corners = 5pt,
    font = \footnotesize,
    draw,
    inner sep = 0pt,
    minimum size = 1.6em
%    minimum width = 1em,
%    minimum height = 1em
  }
  , % nail for distribution
  dot/.style = {
    fill,
    circle,
    inner sep=0mm,
    minimum size=1.25mm,
    line width=0mm
  }
  , % labels of states
  label-state/.style = {
    sloped,
    font = \scriptsize,
    label distance = -2pt
  }
  , % labels of edges
  label-edge/.style = {
    font = \scriptsize,
    label distance = -2pt
  }
}


%------------------------------------------------------------------------------------------------

\newcommand{\powerset}{\mathscr{P}}% requires package mathrsfs
\newcommand{\powersetf}{\powerset_{\mathsf{f}}}

% basic sets
\newcommand{\Prop}{{\rm \sf Prop}\xspace}
\newcommand{\Act}{{\rm \sf Act}\xspace}
\newcommand{\AGT}{{\rm \sf Agt}\xspace}

\newcommand{\FORMS}{{\rm \sf Form}\xspace}

% syntax
\newcommand{\kh}{{\mathsf{Kh}}}
\newcommand{\khi}{{\mathsf{Kh}_i}}
\newcommand{\kc}{{\mathsf{Kc}}}


%\newcommand{\limp}{\rightarrow}
\newcommand{\ra}{\rightarrow}
\newcommand{\lra}{\leftrightarrow}

\newcommand{\liff}{\leftrightarrow}

\newcommand{\A}{{\operatorname{\sf A}}}
\newcommand{\E}{{\operatorname{\sf E}}}

\newcommand{\SAT}{\mathsf{Sat}}

\newcommand{\size}[1]{{|}{#1}{|}}

\newcommand{\Dist}{\mathsf{Dist}}
\newcommand{\support}{\mathsf{Supp}}
\newcommand{\Dirac}{\Delta}
\newcommand{\Cyl}{\mathrm{Cyl}}

\newcommand{\comp}{\circ}


% models
%\newcommand{\modlts}{\mathcal{S}}
\newcommand{\lmodel}{\mathfrak{L}} % lts model
%\newcommand{\umodel}{\mathfrak{M}} % uncertain model
\newcommand{\nmodel}{\mathfrak{N}} % normative model
\newcommand{\model}{\mathfrak{L}}
%\newcommand{\moddults}{\mathcal{D}}
% \newcommand{\cults}{\mathcal{C}}
% \newcommand{\clts}{\mathcal{C}_2}
% \newcommand{\canonical}{\model^\Gamma_c}

\newcommand{\PKh}{\mathcal{L}_{\kh^q}}
\newcommand{\Khlogic}{\mathcal{L}_{\kh}}
\newcommand{\Khunc}{\mathcal{L}_{\kh}}
\newcommand{\PKhunc}{\mathcal{L}^{\mathrm{U}}_{\kh^q}}
\newcommand{\PKhadapt}{\mathcal{L}^{\mathrm{a}}_{\kh^q}}

\newcommand{\fgetprob}{{\normalfont\textsf{fp}}}

% \newcommand{\R}{\operatorname{R}}
\renewcommand{\S}{\operatorname{S}}
\newcommand{\Unc}{\operatorname{U}}
\newcommand{\V}{\operatorname{V}}


\newcommand{\plan}{\pi}
\newcommand{\plans}{\Pi}
\newcommand{\PLANS}{\ACT^{*}}

\newcommand{\reach}[1]{\xrightarrow{#1}}

\newcommand{\complete}{\bot}
\newcommand{\exec}{\mathit{Exec}}
\newcommand{\cexec}{\mathit{CExec}}
\newcommand{\execf}{\exec_{\mathsf{f}}}
\newcommand{\execw}{\exec_\omega}
\newcommand{\cexecf}{\cexec_{\mathsf{f}}}

\newcommand{\Succ}{\mathit{Succ}}

\newcommand{\Comp}{\mathit{Comp}}

\newcommand{\strat}{\sigma}
\newcommand{\D}[1]{\operatorname{D}_{#1}}
\newcommand{\DS}[1]{\operatorname{D}_{#1}}


% macros for examples

\newcommand{\actionfont}{\mathit}
\newcommand{\lift}{\actionfont{lf}}
\newcommand{\stairs}{\actionfont{st}}
\newcommand{\ramp}{\actionfont{rm}}
\newcommand{\panic}{\actionfont{pn}}
\newcommand{\mobile}{\actionfont{mb}}

\newcommand{\init}{\text{\textcolor{blue!60!black}{\normalfont\textsf{init}}}}
\newcommand{\fin}{\text{\textcolor{green!60!black}{\normalfont\textsf{fin}}}}
\newcommand{\goal}{\text{\textcolor{green!60!black}{\ding{51}}}}
\newcommand{\fail}{\text{\textcolor{red!80!black}{\ding{55}}}}

\newcommand{\modelex}{\ensuremath{\model_{\mathrm{e}}}}

%%

\newcommand{\lts}{\textup{LTS}\xspace}
%\newcommand{\ublts}{\textup{LTS}\xspace}
\newcommand{\nts}{\textup{NLTS}\xspace}

\newcommand{\truthset}[2]{\llbracket #2 \rrbracket^{#1}}

%\newcommand{\cmodel}{\modults^\Gamma}


\newcommand{\iffdef}{\ensuremath{\mbox{\it iff}_{\mbox{\tiny\it  def}}}}

% \newcommand{\sel}{\mathsf{sel}}
% \newcommand{\proj}{\mathsf{pr}}

% notions of executability
\newcommand{\stexec}{\operatorname{SE}}

\newcommand{\last}{\mathrm{last}}
\newcommand{\first}{\mathrm{first}}
\newcommand{\pref}{\mathrm{pref}}
\newcommand{\infim}{\mathrm{inf}}
\newcommand{\Prob}{\mathbb{P}}



% axiom systems
% \newcommand{\axm}[1]{{\rm \textsf{#1}}}
% % \newcommand{\KHaxiom}{\ensuremath{\mathcal{L}^{\lts}_{\kh}}\xspace}
% % \newcommand{\KHiaxiom}{\ensuremath{\mathcal{L}^{\ults}_{\khi}}\xspace}
% % \newcommand{\KCiaxiom}{\ensuremath{\mathcal{L}^{\nts}_{\kci,\obliged,\ability}}\xspace}
% \newcommand{\axset}{\mathcal{DL}}
% \newcommand{\kcaxiom}{\ensuremath{\mathcal{DLK}c}\xspace}


% completeness proof
% \newcommand{\smcs}{\boldsymbol{\Phi}}
% % \newcommand{\restkh}[1]{#1\vert_{\kh}}
% % \newcommand{\restnkh}[1]{#1\vert_{\lnot\kh}}
% % \newcommand{\restkhi}[1]{#1\vert_{\khi}}
% % \newcommand{\restnkhi}[1]{#1\vert_{\lnot\khi}}
% \newcommand{\restkc}[1]{#1|_{{\kc}}}
% \newcommand{\restnkc}[1]{#1|_{{\lnot\kc}}}
% \newcommand{\restkci}[1]{#1|_{{\kci}}}
% \newcommand{\restnkci}[1]{#1|_{{\lnot\kci}}}
% \newcommand{\restn}[1]{#1|_{{\normed}}}
% \newcommand{\restnn}[1]{#1|_{{\lnot\normed}}}
% \newcommand{\rests}[1]{#1|_{{\ability}}}
% \newcommand{\restns}[1]{#1|_{{\lnot\ability}}}
% \newcommand{\resta}[1]{#1|_{{\A}}}
% \newcommand{\restna}[1]{#1|_{{\lnot\A}}}

% \newcommand{\restarbitrary}[1]{#1|_{\arbitrary}}
% \newcommand{\restnarbitrary}[1]{#1|_{\lnot\arbitrary}}

% \newcommand{\resta}[1]{#1\vert_{\A}}
% \newcommand{\restna}[1]{#1\vert_{\lnot\A}}

%------------------------------------------------------------------------------------------------

% utils
\newcommand{\card}[1]{{\mid}{#1}{\mid}}
\newcommand{\tup}[1]{\langle{#1}\rangle}
\newcommand{\cset}[1]{\{{#1}\}}
\newcommand{\csetc}[3]{\{ #1 \in #2 \mid #3 \}}
\newcommand{\csetsc}[2]{\{{#1}\mid {#2}\}}
\newcommand{\setof}[2]{\{{#1}\mid {#2}\}}
\newcommand{\set}[1]{\{{#1}\}}

% \newcommand\SetSymbol[1][]{\nonscript\:#1\vert\allowbreak\nonscript\:\mathopen{}}
% \providecommand\given{} % to make it exist
% \DeclarePairedDelimiterX\Set[1]{\left\{}{\right\}}{\renewcommand\given{\SetSymbol[\delimsize]}#1}


\newcommand{\subformulas}{\mathsf{sf}}
\newcommand{\dmd}{dmd}

% \NewDocumentCommand{\setargs}{>{\SplitArgument{1}{;}}m}
% {\setargsaux#1}
% \NewDocumentCommand{\setargsaux}{mm}
% {\IfNoValueTF{#2}{#1} {#1\nonscript\:\delimsize\vert\allowbreak\nonscript\:\mathopen{}#2}}%
% \def\Set{\set*}%

\newcommand{\ssparagraph}[1]{\smallskip\noindent\textbf{#1}\,}

\newenvironment{smallarray}[1]
 {\null\,\vcenter\bgroup\scriptsize
  \renewcommand{\arraystretch}{0.7}%
  \arraycolsep=.13885em
  \hbox\bgroup$\array{@{}#1@{}}}
 {\endarray$\egroup\egroup\,\null}

%------------------------------------------------------------------------------------------------


% To write notes in the text
\newcommand{\raul}[1]{\todo[color=yellow!55]{{\bf Raul:} #1}\xspace}
\newcommand{\bigraul}[1]{\todo[inline,color=yellow!55]{{\bf Raul:} #1}}

\newcommand{\val}[1]{\todo[color=blue!20]{{\bf Val:} #1}\xspace}
\newcommand{\bigval}[1]{\todo[inline,color=blue!20]{{\bf Val:} #1}}

\newcommand{\pedro}[1]{\todo[color=red!20]{{\bf Pedro:} #1}\xspace}
\newcommand{\bigpedro}[1]{\todo[inline,color=red!20]{{\bf Pedro:} #1}}

% \newcommand{\andres}[1]{\todo[color=cyan!20]{{\bf ARS:} #1}\xspace}
% \newcommand{\bigandres}[1]{\todo[inline,color=cyan!20]{{\bf ARS:} #1}}

\newcommand{\pablo}[1]{\todo[color=green!20]{{\bf Pablo:} #1}\xspace}
\newcommand{\bigpablo}[1]{\todo[inline,color=green!20]{{\bf Pablo:} #1}}

\newcommand{\colornuevo}{teal}
%\newcommand{\lineanueva}[1]{\textcolor{red}{#1}}
\newenvironment{textonuevo}
{\color{\colornuevo}}
{\normalcolor}

\newcommand{\colornota}{Peach}
\newenvironment{notas}
{\smallskip \hlight{NOTES:}\;\color{\colornota}}
{\normalcolor}

\newcommand{\colorincompleto}{red}
\newenvironment{incompleto}
{\color{\colorincompleto}}
{\normalcolor}

% \colorlet{colorhighlight}{Yellow}
% \newcommand{\hlight}[1]{{\setlength{\fboxsep}{2pt}\colorbox{colorhighlight}{#1}}}
% \newcommand{\hlightmath}[1]{{\setlength{\fboxsep}{2pt}\colorbox{colorhighlight}{\ensuremath{#1}}}}

%------------------------------------------------------------------------------------------------

% Complexity classes
\newcommand{\NP}{{\rm\textsf{NP}}\xspace}
\newcommand{\CoNP}{{\rm\textsf{Co-NP}}\xspace}

\newcommand{\PTIME}{{\rm\textsf{PTime}}\xspace}
\newcommand{\PSPACE}{{\rm\textsf{PSpace}}\xspace}
\newcommand{\NPSPACE}{{\rm\textsf{NPSpace}}\xspace}
\newcommand{\EXPTIME}{{\rm\textsf{ExpTime}}\xspace}
\newcommand{\NEXPTIME}{{\rm\textsf{NExpTime}}\xspace}
\newcommand{\LSPACE}{{\rm\textsf{LSpace}}\xspace}
\newcommand{\PH}{{\rm\textsf{PH}}\xspace}

%------------------------------------------------------------------------------------------------

\renewcommand{\iff}{\ensuremath{\mathrel{\text{iff}}}}

\newcommand{\tset}[1]{\llbracket #1 \rrbracket}

% \newcommand{\zerodisplayskips}{%
%   \setlength{\abovedisplayskip}{5pt}%
%   \setlength{\belowdisplayskip}{5pt}%
%   \setlength{\abovedisplayshortskip}{5pt}%
%   \setlength{\belowdisplayshortskip}{5pt}}
% \appto{\normalsize}{\zerodisplayskips}
% \appto{\small}{\zerodisplayskips}
% \appto{\footnotesize}{\zerodisplayskips}

% \DeclareMathOperator{\dom}{dom}
% \DeclareMathOperator{\img}{img}
% \DeclareMathOperator{\depth}{\mathsf{md}}
% \DeclareMathOperator{\sforms}{\mathsf{sf}}
% \DeclareMathOperator{\nnf}{nnf}
% \DeclareMathOperator{\cnf}{cnf}
% \DeclareMathOperator{\C}{\mathrm{\Pi}}
% \DeclareMathOperator{\even}{even}
% \DeclareMathOperator{\odd}{odd}
% \DeclareMathOperator{\zeros}{zeros}

\newcommand{\CPL}{\ensuremath{\mathsf{CPL}}}

% \providecommand{\lxor}{\oplus}%{\veebar}
