\documentclass{article}

\usepackage{settings}

\newtheorem{theorem}{Theorem}
\newtheorem{proposition}{Proposition}%[section]
\newtheorem{corollary}{Corollary}%[section]
\newtheorem{lemma}{Lemma}%[section]
\newtheorem{fact}{Fact}%[section]
\newtheorem{remark}{Remark}%[section]
\newtheorem{claim}{Claim}%[section]

\newtheorem{definition}{Definition}%[section]
\newtheorem{example}{Example}%[section]

%------------------------------------------------------------------------------------------------

\usepackage{cleveref}

% Abbreviated Cref
\Crefname{algorithm}{Alg.}{Algs.}
\Crefname{definition}{Def.}{Defs.}
% \Crefname{equation}{Eq.}{Eqs.}
% \Crefname{fact}{Fact}{Facts}
\Crefname{figure}{Fig.}{Figs.}
\Crefname{proposition}{Prop.}{Props.}
% \Crefname{lemma}{Lemma}{Lemmas}
\Crefname{theorem}{Thm.}{Thms.}
\Crefname{example}{Ex.}{Exs.}
\Crefname{corollary}{Cor.}{Cors.}
% \Crefname{enumi}{Item}{Items}
\Crefname{section}{Sec.}{Secs.}
\Crefname{appendix}{App.}{Apps.}
% \Crefname{table}{Table}{Tables}
% \Crefname{inlineenumi}{Item}{Items}
% % \Crefname{cond-khi}{}{}
% % \Crefname{inline-cond-khi}{}{}
% \Crefname{table}{Tab.}{Tabs.}



%% \usepackage{tikz}
%% \usetikzlibrary{arrows,decorations,shapes,automata,positioning,decorations.pathmorphing,patterns}

\usepackage{tikz, tikzscale, pgfplots}
\usetikzlibrary{arrows,arrows.meta,calc,automata,positioning,decorations.pathreplacing,shapes.geometric,shapes.misc,graphs,backgrounds,shadows.blur,snakes}

\tikzset{align at top/.style={baseline=(current bounding box.north)}}
\tikzstyle{every node}=[font=\scriptsize]

\tikzset{
  every picture/.style = {
    thick,
    ->,
    >=stealth',
    node distance = 1.5em and 3em,
  }
  ,
  cross line/.style = {
    preaction = {
      draw=white,
      -,
      line width=4pt
    }
  }
  , % states
  state/.style = {
    circle,
%    rectangle,
%    rounded corners = 5pt,
    font = \footnotesize,
    draw,
    inner sep = 0pt,
    minimum size = 1.6em
%    minimum width = 1em,
%    minimum height = 1em
  }
  , % nail for distribution
  dot/.style = {
    fill,
    circle,
    inner sep=0mm,
    minimum size=1.25mm,
    line width=0mm
  }
  , % labels of states
  label-state/.style = {
    sloped,
    font = \scriptsize,
    label distance = -2pt
  }
  , % labels of edges
  label-edge/.style = {
    font = \scriptsize,
    label distance = -2pt
  }
}


%------------------------------------------------------------------------------------------------

\newcommand{\powerset}{\mathscr{P}}% requires package mathrsfs
\newcommand{\powersetf}{\powerset_{\mathsf{f}}}

% basic sets
\newcommand{\Prop}{{\rm \sf Prop}\xspace}
\newcommand{\Act}{{\rm \sf Act}\xspace}
\newcommand{\AGT}{{\rm \sf Agt}\xspace}

\newcommand{\FORMS}{{\rm \sf Form}\xspace}

% syntax
\newcommand{\kh}{{\mathsf{Kh}}}
\newcommand{\khi}{{\mathsf{Kh}_i}}
\newcommand{\kc}{{\mathsf{Kc}}}


%\newcommand{\limp}{\rightarrow}
\newcommand{\ra}{\rightarrow}
\newcommand{\lra}{\leftrightarrow}

\newcommand{\liff}{\leftrightarrow}

\newcommand{\A}{{\operatorname{\sf A}}}
\newcommand{\E}{{\operatorname{\sf E}}}

\newcommand{\SAT}{\mathsf{Sat}}

\newcommand{\size}[1]{{|}{#1}{|}}

\newcommand{\Dist}{\mathsf{Dist}}
\newcommand{\support}{\mathsf{Supp}}
\newcommand{\Dirac}{\Delta}
\newcommand{\Cyl}{\mathrm{Cyl}}

\newcommand{\comp}{\circ}


% models
%\newcommand{\modlts}{\mathcal{S}}
% \newcommand{\lmodel}{\mathfrak{L}} % lts model
%\newcommand{\umodel}{\mathfrak{M}} % uncertain model
% \newcommand{\nmodel}{\mathfrak{N}} % normative model
\newcommand{\model}{\mathfrak{M}}
%\newcommand{\moddults}{\mathcal{D}}
% \newcommand{\cults}{\mathcal{C}}
% \newcommand{\clts}{\mathcal{C}_2}
% \newcommand{\canonical}{\model^\Gamma_c}

\newcommand{\LogicLetter}{\mathcal{L}}
\newcommand{\PKh}{{\LogicLetter_{\kh^q}}}
\newcommand{\Khlogic}{{\LogicLetter_{\kh}}}
\newcommand{\Khunc}{{\LogicLetter^{\mathrm{U}}_{\kh}}}
\newcommand{\PKhunc}{{\LogicLetter^{\mathrm{U}}_{\kh^q}}}
\newcommand{\PKhadapt}{{\LogicLetter^{\mathrm{a}}_{\kh^q}}}

\newcommand{\fgetprob}{{\normalfont\textsf{rp}}}

% \newcommand{\R}{\operatorname{R}}
\renewcommand{\S}{\operatorname{S}}
\newcommand{\Unc}{\operatorname{U}}
\newcommand{\V}{\operatorname{V}}


\newcommand{\plan}{\pi}
\newcommand{\plans}{\Pi}
\newcommand{\PLANS}{\ACT^{*}}

\newcommand{\reach}[1]{\xrightarrow{#1}}

\newcommand{\complete}{\bot}
\newcommand{\exec}{\mathit{Exec}}
\newcommand{\cexec}{\mathit{CExec}}
\newcommand{\execf}{\exec_{\mathsf{f}}}
\newcommand{\execw}{\exec_\omega}
\newcommand{\cexecf}{\cexec_{\mathsf{f}}}

\newcommand{\Succ}{\mathit{Succ}}

\newcommand{\Comp}{\mathit{Comp}}

\newcommand{\strat}{\sigma}
\newcommand{\D}[1]{\operatorname{D}_{#1}}
\newcommand{\DS}[1]{\operatorname{D}_{#1}}


% macros for examples

\newcommand{\actionfont}{\mathit}
\newcommand{\lift}{\actionfont{lf}}
\newcommand{\stairs}{\actionfont{st}}
\newcommand{\ramp}{\actionfont{rm}}
\newcommand{\panic}{\actionfont{pn}}
\newcommand{\mobile}{\actionfont{mb}}

\newcommand{\init}{\text{\textcolor{blue!60!black}{\normalfont\textsf{init}}}}
\newcommand{\fin}{\text{\textcolor{green!60!black}{\normalfont\textsf{fin}}}}
\newcommand{\goal}{\text{\textcolor{green!60!black}{\ding{51}}}}
\newcommand{\fail}{\text{\textcolor{red!80!black}{\ding{55}}}}

\newcommand{\modelex}{\ensuremath{\model_{\mathrm{e}}}}

%%

\newcommand{\lts}{\textup{LTS}\xspace}
%\newcommand{\ublts}{\textup{LTS}\xspace}
\newcommand{\nts}{\textup{NLTS}\xspace}

\newcommand{\truthset}[2]{\llbracket #2 \rrbracket^{#1}}

%\newcommand{\cmodel}{\modults^\Gamma}


\newcommand{\iffdef}{\ensuremath{\mbox{\it iff}_{\mbox{\tiny\it  def}}}}

% \newcommand{\sel}{\mathsf{sel}}
% \newcommand{\proj}{\mathsf{pr}}

% notions of executability
\newcommand{\stexec}{\operatorname{SE}}

\newcommand{\last}{\mathrm{last}}
\newcommand{\first}{\mathrm{first}}
\newcommand{\pref}{\mathrm{pref}}
\newcommand{\infim}{\mathrm{inf}}
\newcommand{\Prob}{\mathbb{P}}



% axiom systems
% \newcommand{\axm}[1]{{\rm \textsf{#1}}}
% % \newcommand{\KHaxiom}{\ensuremath{\mathcal{L}^{\lts}_{\kh}}\xspace}
% % \newcommand{\KHiaxiom}{\ensuremath{\mathcal{L}^{\ults}_{\khi}}\xspace}
% % \newcommand{\KCiaxiom}{\ensuremath{\mathcal{L}^{\nts}_{\kci,\obliged,\ability}}\xspace}
% \newcommand{\axset}{\mathcal{DL}}
% \newcommand{\kcaxiom}{\ensuremath{\mathcal{DLK}c}\xspace}


% completeness proof
% \newcommand{\smcs}{\boldsymbol{\Phi}}
% % \newcommand{\restkh}[1]{#1\vert_{\kh}}
% % \newcommand{\restnkh}[1]{#1\vert_{\lnot\kh}}
% % \newcommand{\restkhi}[1]{#1\vert_{\khi}}
% % \newcommand{\restnkhi}[1]{#1\vert_{\lnot\khi}}
% \newcommand{\restkc}[1]{#1|_{{\kc}}}
% \newcommand{\restnkc}[1]{#1|_{{\lnot\kc}}}
% \newcommand{\restkci}[1]{#1|_{{\kci}}}
% \newcommand{\restnkci}[1]{#1|_{{\lnot\kci}}}
% \newcommand{\restn}[1]{#1|_{{\normed}}}
% \newcommand{\restnn}[1]{#1|_{{\lnot\normed}}}
% \newcommand{\rests}[1]{#1|_{{\ability}}}
% \newcommand{\restns}[1]{#1|_{{\lnot\ability}}}
% \newcommand{\resta}[1]{#1|_{{\A}}}
% \newcommand{\restna}[1]{#1|_{{\lnot\A}}}

% \newcommand{\restarbitrary}[1]{#1|_{\arbitrary}}
% \newcommand{\restnarbitrary}[1]{#1|_{\lnot\arbitrary}}

% \newcommand{\resta}[1]{#1\vert_{\A}}
% \newcommand{\restna}[1]{#1\vert_{\lnot\A}}

%------------------------------------------------------------------------------------------------

% utils
\newcommand{\card}[1]{{\mid}{#1}{\mid}}
\newcommand{\tup}[1]{\langle{#1}\rangle}
\newcommand{\cset}[1]{\{{#1}\}}
\newcommand{\csetc}[3]{\{ #1 \in #2 \mid #3 \}}
\newcommand{\csetsc}[2]{\{{#1}\mid {#2}\}}
\newcommand{\setof}[2]{\{{#1}\mid {#2}\}}
\newcommand{\set}[1]{\{{#1}\}}

% \newcommand\SetSymbol[1][]{\nonscript\:#1\vert\allowbreak\nonscript\:\mathopen{}}
% \providecommand\given{} % to make it exist
% \DeclarePairedDelimiterX\Set[1]{\left\{}{\right\}}{\renewcommand\given{\SetSymbol[\delimsize]}#1}


\newcommand{\subformulas}{\mathsf{sf}}
\newcommand{\dmd}{dmd}

% \NewDocumentCommand{\setargs}{>{\SplitArgument{1}{;}}m}
% {\setargsaux#1}
% \NewDocumentCommand{\setargsaux}{mm}
% {\IfNoValueTF{#2}{#1} {#1\nonscript\:\delimsize\vert\allowbreak\nonscript\:\mathopen{}#2}}%
% \def\Set{\set*}%

\newcommand{\ssparagraph}[1]{\smallskip\noindent\textbf{#1}\,}

\newenvironment{smallarray}[1]
 {\null\,\vcenter\bgroup\scriptsize
  \renewcommand{\arraystretch}{0.7}%
  \arraycolsep=.13885em
  \hbox\bgroup$\array{@{}#1@{}}}
 {\endarray$\egroup\egroup\,\null}

%------------------------------------------------------------------------------------------------


% To write notes in the text
\newcommand{\raul}[1]{\todo[color=yellow!55]{{\bf Raul:} #1}\xspace}
\newcommand{\bigraul}[1]{\todo[inline,color=yellow!55]{{\bf Raul:} #1}}

\newcommand{\val}[1]{\todo[color=blue!20]{{\bf Val:} #1}\xspace}
\newcommand{\bigval}[1]{\todo[inline,color=blue!20]{{\bf Val:} #1}}

\newcommand{\pedro}[1]{\todo[color=red!20]{{\bf Pedro:} #1}\xspace}
\newcommand{\bigpedro}[1]{\todo[inline,color=red!20]{{\bf Pedro:} #1}}

% \newcommand{\andres}[1]{\todo[color=cyan!20]{{\bf ARS:} #1}\xspace}
% \newcommand{\bigandres}[1]{\todo[inline,color=cyan!20]{{\bf ARS:} #1}}

\newcommand{\pablo}[1]{\todo[color=green!20]{{\bf Pablo:} #1}\xspace}
\newcommand{\bigpablo}[1]{\todo[inline,color=green!20]{{\bf Pablo:} #1}}

\newcommand{\colornuevo}{teal}
%\newcommand{\lineanueva}[1]{\textcolor{red}{#1}}
\newenvironment{textonuevo}
{\color{\colornuevo}}
{\normalcolor}

\newcommand{\colornota}{Peach}
\newenvironment{notas}
{\smallskip \hlight{NOTES:}\;\color{\colornota}}
{\normalcolor}

\newcommand{\colorincompleto}{red}
\newenvironment{incompleto}
{\color{\colorincompleto}}
{\normalcolor}

% \colorlet{colorhighlight}{Yellow}
% \newcommand{\hlight}[1]{{\setlength{\fboxsep}{2pt}\colorbox{colorhighlight}{#1}}}
% \newcommand{\hlightmath}[1]{{\setlength{\fboxsep}{2pt}\colorbox{colorhighlight}{\ensuremath{#1}}}}

%------------------------------------------------------------------------------------------------

% Complexity classes
\newcommand{\NP}{{\rm\textsf{NP}}\xspace}
\newcommand{\CoNP}{{\rm\textsf{Co-NP}}\xspace}

\newcommand{\PTIME}{{\rm\textsf{PTime}}\xspace}
\newcommand{\PSPACE}{{\rm\textsf{PSpace}}\xspace}
\newcommand{\NPSPACE}{{\rm\textsf{NPSpace}}\xspace}
\newcommand{\EXPTIME}{{\rm\textsf{ExpTime}}\xspace}
\newcommand{\NEXPTIME}{{\rm\textsf{NExpTime}}\xspace}
\newcommand{\LSPACE}{{\rm\textsf{LSpace}}\xspace}
\newcommand{\PH}{{\rm\textsf{PH}}\xspace}

%------------------------------------------------------------------------------------------------

\renewcommand{\iff}{\ensuremath{\mathrel{\text{iff}}}}

\newcommand{\tset}[1]{\llbracket #1 \rrbracket}

% \newcommand{\zerodisplayskips}{%
%   \setlength{\abovedisplayskip}{5pt}%
%   \setlength{\belowdisplayskip}{5pt}%
%   \setlength{\abovedisplayshortskip}{5pt}%
%   \setlength{\belowdisplayshortskip}{5pt}}
% \appto{\normalsize}{\zerodisplayskips}
% \appto{\small}{\zerodisplayskips}
% \appto{\footnotesize}{\zerodisplayskips}

% \DeclareMathOperator{\dom}{dom}
% \DeclareMathOperator{\img}{img}
% \DeclareMathOperator{\depth}{\mathsf{md}}
% \DeclareMathOperator{\sforms}{\mathsf{sf}}
% \DeclareMathOperator{\nnf}{nnf}
% \DeclareMathOperator{\cnf}{cnf}
% \DeclareMathOperator{\C}{\mathrm{\Pi}}
% \DeclareMathOperator{\even}{even}
% \DeclareMathOperator{\odd}{odd}
% \DeclareMathOperator{\zeros}{zeros}

\newcommand{\CPL}{\ensuremath{\mathsf{CPL}}}

% \providecommand{\lxor}{\oplus}%{\veebar}


\begin{document}

\title{How lucky are you to know your way}
\author{Valentin Cassano, Pablo F. Castro, Pedro R. D'Argenio and Raul Fervari}
\date{}


\maketitle

\begin{abstract}
    In this paper we introduce a probabilistic version of knowing how modalities. More precisely, our logics extend existing approaches to model the ability of an agent to achieve a goal, with a certain probability. Models of the logic are extended with probability distributions over the actions that can be executed. In turn, we investigate different variant logics with this new feature. First, we enrich with probabilities the linear plan-based logic of knowing how. Then, we consider indistinguishability classes, and obtain two logics, one that we call with `non-adaptable' plans, and another called with `adaptable' plans. For all these logics we investigate the computational complexity of their model-checking problem, obtaining uncedidability results for the first and the second logic, while for the last one the problem is decidable in polynomial time.
\end{abstract}

\section{Introduction}
\label{sec:intro}

Modern approaches in Epistemic Logic~\cite{Hintikka:1962} have shifted focus from a single notion of knowledge (usually, the notion of \emph{knowing that}) to a diverse palette of notions, each of them tailored to specific purposes. In this regard, the notion of \emph{knowing how} has received significant attention, since it captures scenarios related to intelligent agents and its strategic behaviour. Logically, knowing how is typically defined as the ability of an agent to achieve a certain goal.  Knowing-how logics have direct applications to \emph{planning} problems \cite{Stuart21},  where  plans have to be constructed such that a collection of agents can achieve a given goal.  Examples of planning applications can be found in e.g. self driving cars,  robotics,  conversational agents,  cybersecurity,  risk management,  etc.

Usually,  the semantics for knowing-how logics can be thought as a combination of operators that describe abilities alongside standard epistemic operators for knowing that.  This is the approach introduced in, e.g.~\cite{Mccarthy69,Moore85,Les00,Hoek00,HerzigT06}. As a result,  knowing how reflects that \emph{the agent knows that there is a course of action leading to achieve the intended goal}. However,  as pointed out in e.g.~\cite{JamrogaA07,Herzig15}, this reading is not entirely accurate.  Instead, knowing how could be read as \emph{there is a course of action, that the agent knows how to apply, to bring about the goal}. Thus, a novel perspective emerged in~\cite{Wang15lori,Wang16,Wang2016}, where a new modality is defined with the aim of capturing the proper reading of knowing how. 

More specifically, the new modality under consideration is a binary modality $\kh(\psi,\varphi)$ interpreted over Labeled Transition Systems (LTS), where an LTS models the  actions that are available to the agents as well as the effects of these actions. Thus,  the formula $\kh(\psi,\varphi)$ holds if there exists a sequence of actions $\plan$ (i.e., a \emph{plan}) such that,  in every situation where $\psi$ holds, $\plan$ can be executed, it never aborts its execution, and it always leads to situations in which $\varphi$ holds. This new view of knowing how raised a new family of logics refining the original one, witnessed by the extensive related literature (see e.g.~\cite{LiWang17,Li17,FervariHLW17,NaumovT19,Naumov2018a}). Interestingly, in~\cite{AFSVQ21,AFSVQ23} the original approach is enriched by a notion of `epistemic indistinguishability' between plans, arguably closer to standard semantics of epistemic logics. This indistinguishability relation indicates that all the related plans are considered or perceived ``equally good''
from the agent's perspective (even if they are not), thus a plan $\plan$ is suitable for achieving a goal if this is also the case for all the plans that are indistinguishable from $\plan$. The new semantics arguably offers a more adequate view of knowing how from an epistemic perspective, compared to the original approach.

The above-mentioned works investigate various logical properties, including axiomatizations, expressivity, and the complexity of the respective logics. In particular, the model-checking problem results of interest, since it is  ubiquitous in software verification but also an important tool for controller synthesis and planning. Moreover, as argued in~\cite{DF23}, the model-checking problem better reflects the real power of the logics. This is because these logics often have a simple syntax combined with a rich semantics. In model-checking, plans are part of the input, so the complexity needs to be tamed (unlike in the satisfiability problem, where other tricks can be used like guessing a proper plan). Therein,  the authors also discuss how to incorporate other constraints into the plans, more precisely, the different semantics of knowing how modalities are enriched with regularity constraints (i.e., where plans are given by some regular formalism) and numerical constraints (i.e., where actions in knowing how are restricted by some budget). 

The work in~\cite{DF23} paves the way for studying additional constraints, particularly knowing how to achieve a goal with a certain probability. This is of interest in scenarios where plans might lead to unexpected results due to a faulty behavior of actions,  or because their executions lead to random outcomes. Just as (constrained) planning is connected to (constrained) knowing-how, the ability to handle probabilities in the context of knowing how serves as the logical counterpart to probabilistic planning (see, e.g.,~\cite{MadaniHC99}) and related concepts. Moreover, there is a realm of logics featuring probabilistic notions of strategic reasoning. For instance, \cite{BaierAFK18} discusses the idea of model-checking with probabilities, while those specifically related to ATL are explored in~\cite{BA95,TJ07,BullingJ09}. Also, probabilistic strategy logics are investigated in~\cite{AKMM19}, while~\cite{BerthonKMM24} considers stochastic natural strategic abilities. % just to name a few.

Here, we present extensions of knowing how logics with probabilities. The new modality, written $\kh_q(\psi,\varphi)$, will be read as \emph{the agent knows how to achieve $\varphi$ given $\psi$, with a probability of at least $q$}. This idea results helpful in modeling case studies where the result of actions have a random component. A simple example of this situation is given in \cite{Kushmerick1995}: consider a robot that has to grasp an object,  the result of the robot's actions stochastically depends on the state of the world. For instance,  if the gripper is wet,  there is  a probability of $0.9$ that the object falls when the robot tries to pick it up.  Thus,  the robot may try to dry the gripper before picking up elements.   These kinds of scenarios can be modeled with Probabilistic LTS (PLTS), i.e., transition systems where now transitions relate states with probability distributions,  which in turn  capture the stochastic behavior of actions.

We start in~\Cref{subsec:prob:linear} by naturally extending the logic over linear plans from~\cite{Wang15lori,Wang16,Wang2016}.  For this logic we show that, under non-probabilistic models (i.e., LTSs), it agrees with the non-probabilistic case. We prove that the model-checking problem for the new logic is undecidable,  which contrasts  with the $\PSPACE$-complete complexity of the base logic. The proof strategy relies on the emptiness problem for probabilistic automata~\cite{MadaniHC99}. Then, we extend with probabilities the knowing-how logic over LTS with indistinguishability classes, for which we have devised two cases.  First (\Cref{subsec:prob:indist:committed}), we directly add probabilities to the logic presented in~\cite{AFSVQ21}.  In this case,  that we call \emph{non-adaptative}, the model-checking problem becomes undecidable  contrasting the complexity of model checking the original logic, which is in  $\PTIME$. This is proven by using a variant of the result in~\cite{MadaniHC99}. The second proposal (\Cref{subsec:prob:indist:adaptive}), called here \emph{adaptative}, is arguably suitable to model scenarios in which the agent has the ability to choose between one plan or another, among those that are considered equally good, depending on the particular situation. In fact, we compare expressivenes of all three logics with indistinguishability which helps to understand their utility. Also, for the adaptive case, 
we get that the model-checking problem is in $\PTIME$, another appealing characteristic of this logic. Along the paper, we discuss a running example to illustrate not only the behaviour of the logics, but also our design decisions.  
% \emph{Omitted proofs are included in the supplementary material.}
% Finally, we  have implemented the algorithms for the decidable cases into PRISM tool,  the prototype is presented in Section \ref{} along with its application to some case studies. 
\iffalse
---------- 

Here we list some important pieces of work, motivating ours:

\begin{itemize}
    % \item Knowing how has been investigated is the last years, from different perspectives, especially by combining epistemic operators of knowing that with operators describing abilities~\cite{Mccarthy69,Moore85,Les00,Hoek00,HerzigT06}. This is not considered as a proper reading~\cite{JamrogaA07,Herzig15}
    % \item In~\cite{Wang15lori,Wang16,Wang2016} a new perspective on knowing how emerged, in which a new modality is specifically defined with the purpose of capturing this concept.
    % \item This raised a family of logics, witnessed by all related work (see e.g.~\cite{LiWang17,Li17,Li17bis,FervariHLW17,LiW21,NaumovT17,NaumovT18,NaumovT19,Naumov2018a}).
    % \item A notion of `epistemic indistinguishability' is missing, arguably fixed by~\cite{AFSVQ21,AFSVQ23}.
    % \item With this at hand, it was possible to define dynamic epistemic modalities (e.g.~\cite{AFSV22}).
    % \item Constraints on plans, like regularity or budget constraints~\cite{DemriF23}.
    % \item The latter opens the path to study other constraints, in particular, \emph{knowing how to achieve a goal with a certain probability}.
    \item Relate to other epistemic based logics with probabilities, and with the version of knowing how with uncertainty~\cite{NaumovT19}. Recall the differences, and the case of use that we are able to capture.
    % \item Relate to planning with probabilities.
    % \item Model-checking with probabilities \cite{BaierAFK18}, related to ATL \cite{BA95,TJ07,BullingJ09}, strategy logics \cite{AKMM19}, stochastic natural strategic abilities \cite{BerthonKMM24}
    \item Recall the different versions of our modality, how we obtain a decidable logic, and why it makes sense.
    \item Connections with reinforcement learning and reasoning about such scenarios.
\end{itemize}
\fi


\section{Preliminaries}
\label{sec:preliminaries}

Here we introduce some definitions, extending those in e.g.~\cite{AFSVQ21}.

\begin{definition}\label{def:plts}
    Let $\PROP$ be a countable set of propositional symbols and let $\AGT$ be a finite set of agents.  
    A \emph{Probabilistic Labeled Transition System (PLTS)} $\model$ is a tuple
    $\tup{\S,\ACT,\dist(S),\ra,\sim,\V}$ such that:
    \begin{itemize}
        \item $\S$ is a countable set of states,
        \item $\ACT$ is a set of action symbols,
        \item $\dist(\S)$ is the set of probability distributions over $\S$,
        \item $\ra \subseteq \S \times \ACT \times \dist(\S)$ is a transition relation,
        \item $\sim\subseteq \DS{i} \times \AGT \times \DS{i}$ is an indistinguishability relation between plans for each agent over $\DS{i}\subseteq$, and
        \item $\V: \S \ra 2^\PROP$ is a valuation function.
    \end{itemize}
    We denote by $[\plan]_{\sim_i}:=\set{\plan' \in \DS{i} \mid \plan \sim_i \plan'}$ is $\plan$'s equivalence relation with respect to $\DS{i}$  and define $\Unc(i) := \set{[\plan]_{\sim_i} \mid \plan\in\DS{i}}$. The set $\Unc:=\set{\Unc(i) \mid i\in\AGT}$ is called the  \emph{uncertainty set} of $\model$. For simplicity sake, we sometimes denote $\model=\tup{\S,\ACT,\dist(\S),\ra,\Unc,\V}$ to refer to a PLTS, i.e., we will use its uncertainty set instead of the indistinguishability relation.
\end{definition}

\begin{definition}
    \label{def:syntax}
    The set of formulas (a.k.a. the language) of $\PKh$ is defined by the following BNF:
    \[
        \varphi, \psi ::= p \mid \neg \varphi \mid \varphi \vee \psi \mid \kh_i^\rho(\psi,\varphi),
    \]
    where $p\in\PROP$, $i\in\AGT$ and $\rho\in[0,1]$. Other Boolean operators are defined as usual. Formulas of the form $\kh_i^\rho(\psi,\varphi)$ are read as \emph{``agent $i$ knows how to achieve $\varphi$ given $\psi$, with probability $\rho$''}
\end{definition}

\begin{definition} \label{def:semantics}
    Let $\model = \tup{\S,\ACT,\dist(\S),\ra,\Unc,\V}$ be a PLTS and let $s\in\S$, the satisfiability relation $\models$ for $\PKh$ is inductively defined as:
    \[
    \begin{array}{l@{\ \ \ }c@{\ \ \  }l}
    \model, s \models p & \iffdef & p \in \V(s) \\
    \model, s\models \neg\varphi & \iffdef & \model, s \not\models \varphi \\
    \model, s \models \psi\vee\varphi & \iffdef & \model, s \models \psi \mbox{ or }\model, w \models \varphi \\
    \model, s \models \kh_i^\rho(\psi,\varphi) & \iffdef & \text{there is } \plans \in \Unc(i) \;\text{such that:} \\
    & & \ \ \text{\rm (1)} \ \plans \text{ is $\rho$-executable at }  \truthset{\model}{\psi}\; \text{and} \\
    & & \ \ \text{\em (2)} \ \truthset{\model}{\psi} \reach{\plans}_\rho \truthset{\model}{\varphi}, 
    \end{array}
    \]     \raul{$\rho$-executable and $ \reach{\plans}_\rho$ to be defined}
    where: $\truthset{\model}{\chi} := \csetsc{s\in\S}{\model,w\models\chi}$. Define: $\model\models\varphi$ iff  $\truthset{\model}{\varphi}=\S$, and $\models\varphi$ iff $\model\models\varphi$, for all PLTS $\model$.
\end{definition}
\section{Knowing How with Linear Plans}
\label{sec:khlinearplans}

\subsection{Basic Definitions}

We start by introducing the most basic notion of knowing how as defined in e.g.~\cite{Wang15lori,Wang16,Wang2016}.


\begin{definition}\label{def:lts}
    Let $\PROP$ be a countable set of propositional symbols. 
    A \emph{Labeled Transition System (LTS)}  is a tuple
    $\model=\tup{\S,\ACT,\ra,\V}$ such that:
    \begin{itemize}
        \item $\S$ is a countably non-empty set of states,
        \item $\ACT$ is a countable set of action symbols,
        \item ${\ra} \subseteq \S \times \ACT \times \S$ is a transition relation (sometimes we denote by $\reach{a}$ the set $\set{(s,t) \mid (s,a,t){\in}{\ra}}$, for $a\in\ACT$).
    \end{itemize}
    Elements of $\ACT^*$ are called \emph{plans} (with $\epsilon$ the empty plan).  Let $\plan\in\ACT^*$, $\size{\plan}$ denotes its length ($\size{\epsilon}:=0$).
    For  $0\leq i \leq \size{\plan}$, the plan $\plan_i$ denotes the initial segment of $\plan$ up to (and including) the $i^{th}$ position (with $\plan_0 := \epsilon$). The action $\plan[i]$ is the one appearing in $\plan$ at the $i^{th}$ position. We define $\reach{\plan}$ as the composition $\reach{\plan[1]} \comp \ldots \comp \reach{\plan[\size{\plan}]}$. 

    We say that a plan $\plan\in\ACT^*$ is \emph{strongly executable (SE)} at a state $s\in\S$ if and only if, for all $0\leq i \leq \size{\plan}-1$, for all $t\in\S$ such that $s\reach{\plan_i} t$, there is $v\in\S$ such that $t\reach{\plan[i+1]} v$. The plan $\plan$ is SE at $T\subseteq \S$ if and only if it is SE at every $s\in T$.
\end{definition}

\begin{definition}
    \label{def:syntax}
    The set of formulas (a.k.a. the language) of $\Khlogic$ is defined by the following BNF:
    \[
        \varphi, \psi ::= p \mid \neg \varphi \mid \varphi \vee \psi \mid \kh(\psi,\varphi),
    \]
    where $p\in\PROP$. Other Boolean operators are defined as usual. Formulas of the form $\kh(\psi,\varphi)$ are read as \emph{``the agent knows how to achieve $\varphi$ given $\psi$''}
\end{definition}

\begin{definition} \label{def:semantics-kh}
    Let $\model = \tup{\S,\ACT,\ra,\V}$ be an LTS and let $s\in\S$, the satisfiability relation $\models$ for $\Khlogic$ is inductively defined as:
    \[
    \begin{array}{l@{\ \ \ }c@{\ \ \  }l}
    \model, s \models p & \iffdef & p \in \V(s) \\
    \model, s\models \neg\varphi & \iffdef & \model, s \not\models \varphi \\
    \model, s \models \psi\vee\varphi & \iffdef & \model, s \models \psi \mbox{ or }\model, w \models \varphi \\
    \model, s \models \kh(\psi,\varphi) & \iffdef & \text{there is } \plan \in \ACT^* \;\text{such that:} \\
    & & \ \ \text{\rm (1)} \ \plan \text{ is SE-executable at }  \truthset{\model}{\psi}\; \text{and} \\
    & & \ \ \text{\em (2)} \ \truthset{\model}{\psi} \reach{\plan} \truthset{\model}{\varphi}, 
    \end{array}
    \]      where: $\truthset{\model}{\chi} := \csetsc{s\in\S}{\model,w\models\chi}$. Define: $\model\models\varphi$ iff  $\truthset{\model}{\varphi}=\S$, and $\models\varphi$ iff $\model\models\varphi$, for all LTS $\model$.
\end{definition}

\subsection{The Probabilistic Approach}

\begin{definition}\label{def:plts}
    Let $\PROP$ be a countable set of propositional symbols. 
    A \emph{Probabilistic Labeled Transition System (PLTS)}  is a tuple
    $\model=\tup{\S,\ACT,\ra,\V}$, defined exactly as an LTS except that ${\ra}\subseteq \S \times \ACT \times \dist(\S)$, where  $\dist(\S)$ is the set of probability distributions over $\S$.
\end{definition}

    \begin{figure}[t]
        \begin{center}
            \includegraphics[scale=0.1]{PLTS.jpg}
        \end{center}
    \end{figure}

\begin{definition}\label{def:strategy-comp-exec}
    Let $\model=\tup{\S,\ACT,\ra,\V}$ be a PLTS, a \emph{strategy} is a function $\strategy: \S\times(\ACT\times\S)^* \ra \dist(\S\times\ACT\times\dist(\S))$. Let $\plan\in\ACT^*$, we say that $\strategy$ is \emph{$\plans$-compatible} if and only if, for all $\rho\in \S\times(\ACT\times\S)^*$, the following conditions hold:
    \begin{enumerate}
        \item $\strategy(\rho)(s,a,\mu)>0$ implies $\last(\rho)=s$ and $s\reach{a}\mu$ in $\model$, and 
        \item $\bar{\rho}\in\pref(\plans)$ implies $\bar{\rho}a\in\pref(\plans)$.
    \end{enumerate}
    For $\plan\in\ACT^*$, we say that $\sigma$ is \emph{$\plan$-compatible} if it is $\{\plan\}$-compatible. 

    Let $q\in[0,1]$, we say that $\plans$ is \emph{$q$-executable} at $s\in\S$, if and only if, 
    \[
        \infim_{\{\strategy \mid \strategy \text{ is } \plans\text{-comp.}\}} \probability^\strategy_{s}(\plans)\geq q.
    \]
    Finally, $\plans$ is \emph{$q$-executable} at $B\subseteq\S$, if and only if, it is $q$-executable at every $s\in B$. 
    A plan $\plan$ is \emph{$q$-executable} at $s$ (respectively, at $B$) if $\{\plan\}$ is $q$-executable at $s$ (respectively, at $B$). 
\end{definition}

\begin{definition}
    Let $\model = \tup{\S,\ACT,\ra,\V}$ be a PLTS and let $s\in\S$, the satisfiability relation $\models$ for $\PKh$ is inductively defined as:
    \[
        \begin{array}{l@{\ \ \ }c@{\ \ \  }l}
        % \model, s \models p & \iffdef & p \in \V(s) \\
        % \model, s\models \neg\varphi & \iffdef & \model, s \not\models \varphi \\
        % \model, s \models \psi\vee\varphi & \iffdef & \model, s \models \psi \mbox{ or }\model, w \models \varphi \\
        \model, s \models \kh^q(\psi,\varphi) & \iffdef & \text{there is } \plan \in \ACT^* \;\text{such that:} \\
        & & \ \ \text{\rm (1)} \ \plan \text{ is $q$-executable at }  \truthset{\model}{\psi}\; \text{and} \\
        & & \ \ \text{\em (2)} \ \truthset{\model}{\psi} \reach{\plan}_q \truthset{\model}{\varphi}, 
        \end{array}
        \] \raul{1 and 2 will be merged into one condition.}
        where: $\truthset{\model}{\chi} := \csetsc{s\in\S}{\model,w\models\chi}$. Define: $\model\models\varphi$ iff  $\truthset{\model}{\varphi}=\S$, and $\models\varphi$ iff $\model\models\varphi$, for all PLTS $\model$.
\end{definition}

\begin{theorem}\label{th:mc-khp-undecidable}
The model-checking problem for $\PKh$ is undecidable.
\end{theorem}

\begin{proof}
    Suppose we want to check whether $\model,s\models\kh^q(\psi,\varphi)$.  W.l.o.g., consider $\model$ is complete and deterministic. 
    From the semantics, we need to check if there is $\plan\in\ACT^*$ satisfying conditions (1) and (2). Consider condition (1), i.e., we need to check whether $\plan$ is $q$-executable at all $s\in\truthset{\model}{\psi}$. Fix such an $s$. Since $\model$ is deterministic and complete, it all boils down to check whether there is $\plan\in\ACT^*$ such that $\probability^\strategy_s(\plan)\geq q$ (notice that $\strategy$ is unique by determinism of $\model$). 
    The latter solves the problem of checking whether the language recognized by a probabilistic (deterministic) automata is empty, which is undecidable~\cite{MadaniHC99}. 
\end{proof}
\section{Knowing How over Indistinguishability classes}
\label{sec:kh:indistinguishability}


\subsection{A First Approach: Non-Adaptable Plans}

Here we introduce some definitions, extending those in e.g.~\cite{AFSVQ21,AFSVQ23}.


\begin{definition}\label{def:plts}
    Let $\PROP$ be a countable set of propositional symbols and let $\AGT$ be a finite set of agents.  
    A \emph{Probabilistic Labeled Transition System with uncertainty (PLTSU)}  is a tuple
    $\model=\tup{\S,\ACT,\ra,\sim,\V}$ s.t. $\tup{\S,\ACT,\ra,\V}$ is a PLTS, and 
    \begin{itemize}
        % \item $\S$ is a countably non-empty set of states,
        % \item $\ACT$ is a countable set of action symbols,
        % \item $\dist(\S)$ is the set of probability distributions over $\S$,
        % \item ${\ra} \subseteq \S \times \ACT \times \dist(\S)$ is a transition relation,
        \item ${\sim}\subseteq \DS{}\times \DS{}$, where $\DS{}\subseteq\ACT^*$,  is an equivalence relation over $\DS{}$, and 
        % \item $\V: \S \ra 2^\PROP$ is a valuation function.
    \end{itemize}
    %Elements of $\ACT^*$ are called \emph{plans}, and 
    Each $\sim$ is called the indistinguishability relation between plan. 
    % We write $\plan\sim_i\plan'$ whenever $(\plan,i,\plan')\in{\sim}$, and we define $\DS{i}:=\set{\plan \mid \text{ there is } \plan' \text{ s.t. } \plan\sim_i\plan'}$. 
    By $[\plan]_{\sim}:=\set{\plan' \in \DS{} \mid \plan \sim_i \plan'}$ we denote $\plan$'s equivalence relation with respect to $\DS{}$, then we define the \emph{indistinguishability set of $\model$} as $\Unc := \set{[\plan]_{\sim} \mid \plan\in\DS{}}$. 
    %The set $\Unc:=\set{\Unc(i) \mid i\in\AGT}$ is called the \emph{uncertainty set} of $\model$. 
    For simplicity sake, we sometimes denote $\model=\tup{\S,\ACT,\ra,\Unc,\V}$ to refer to a PLTS, i.e., we will use its uncertainty set instead of the indistinguishability relation.
\end{definition}

Notice that, as it is defined in~\cite{AFSVQ21,AFSVQ23}, $\Unc$ represents the perception the agent has about the reality. In turn, the relation $\sim$ is not an equivalence relation over $\ACT^*$ but over $\DS{}$, as the latter contains only the plans she considers available or suitable for her purposes, while those in $\ACT^*\setminus\DS{}$ are not considered by the agent, even if they are suitable plans. 

% \begin{definition} \label{def:executability} \raul{TBC}
%     Let  $\model=\tup{\S,\ACT,\dist(\S),\ra,\Unc,\V}$ be a PLTSU, let $\plan\in\ACT^*$ be a plan, and let $q\in[0,1]$ a probability, we need to define two notions:
%     \begin{enumerate}
%         \item We say $\plan$ is \emph{$q$-executable} at $s\in\S$ iff ...  Extend it for set of plans and set of states.
%         \item For $s,t\in\S$, we write $s \reach{\plan}_q t$ iff ... Extend it for set of plans and set of states.
%     \end{enumerate}
% \end{definition}

In general, we will assume that PLTS are complete, as for those we can always complete the execution of a plan to a \emph{sink state}.

% \begin{definition}
%     \label{def:syntax}
%     The set of formulas (a.k.a. the language) of $\PKhunc$ is defined by the following BNF:
%     \[
%         \varphi, \psi ::= p \mid \neg \varphi \mid \varphi \vee \psi \mid \kh_i^q(\psi,\varphi),
%     \]
%     where $p\in\PROP$, $i\in\AGT$ and $q\in[0,1]$. Other Boolean operators are defined as usual. Formulas of the form $\kh_i^q(\psi,\varphi)$ are read as \emph{``agent $i$ knows how to achieve $\varphi$ given $\psi$, with probability $q$''}
% \end{definition}

\begin{definition} \label{def:semantics-non-adap}
    Let $\model = \tup{\S,\ACT,\ra,\Unc,\V}$ be a PLTSU and let $s\in\S$, the satisfiability relation $\models$ for $\PKhunc$ is inductively defined as:
    \[
    \begin{array}{l@{\ \ \ }c@{\ \ \  }l}
    % \model, s \models p & \iffdef & p \in \V(s) \\
    % \model, s\models \neg\varphi & \iffdef & \model, s \not\models \varphi \\
    % \model, s \models \psi\vee\varphi & \iffdef & \model, s \models \psi \mbox{ or }\model, w \models \varphi \\
    \model, s \models \kh^q(\psi,\varphi) & \iffdef & \text{there is } \plans \in \Unc \;\text{such that for all } \plan\in\plans{:} \\
    & & \ \ \text{\rm (1)} \ \plan \text{ is $q$-executable at }  \truthset{\model}{\psi}\; \text{and} \\
    & & \ \ \text{\em (2)} \ \truthset{\model}{\psi} \reach{\plan}_q \truthset{\model}{\varphi}, 
    \end{array}
    \]     \raul{1 and 2 to be merge into one.}
    \noindent where: $\truthset{\model}{\chi} := \csetsc{s\in\S}{\model,w\models\chi}$. Define: $\model\models\varphi$ iff  $\truthset{\model}{\varphi}=\S$, and $\models\varphi$ iff $\model\models\varphi$, for all PLTS $\model$.
\end{definition}

This definition bears a resemblance to the one in Section~\ref{sec:khlinearplans}, except that we ask the conditions hold for \emph{every plan} belonging to a set of indistinguishable plans $\plans$. This way, we obtain a result similar to the one in the previous section.


\subsection{A Second Approach: Adaptable Plans}

\begin{theorem}\label{th:mc-khp-nadapt-undecidable}
    The model-checking problem for $\PKhunc$ under the semantics from Definition~\ref{def:semantics:non-adap} is undecidable.
\end{theorem}

\begin{definition} \label{def:semantics-adap}
    Let $\model = \tup{\S,\ACT,\ra,\Unc,\V}$ be a PLTSU and let $s\in\S$, the satisfiability relation $\models$ for $\PKhunc$ is inductively defined as:
    \[
    \begin{array}{l@{\ \ \ }c@{\ \ \  }l}
    % \model, s \models p & \iffdef & p \in \V(s) \\
    % \model, s\models \neg\varphi & \iffdef & \model, s \not\models \varphi \\
    % \model, s \models \psi\vee\varphi & \iffdef & \model, s \models \psi \mbox{ or }\model, w \models \varphi \\
    \model, s \models \kh^q(\psi,\varphi) & \iffdef & \text{there is } \plans \in \Unc \;\text{such that:} \\
    & & \ \ \text{\rm (1)} \ \plans \text{ is $q$-executable at }  \truthset{\model}{\psi}\; \text{and} \\
    & & \ \ \text{\em (2)} \ \truthset{\model}{\psi} \reach{\plans}_q \truthset{\model}{\varphi}, 
    \end{array}
    \]     \raul{1 and 2 to be merge into one.}
    \noindent where: $\truthset{\model}{\chi} := \csetsc{s\in\S}{\model,w\models\chi}$. Define: $\model\models\varphi$ iff  $\truthset{\model}{\varphi}=\S$, and $\models\varphi$ iff $\model\models\varphi$, for all PLTS $\model$.
\end{definition}

The general idea in condition 1 of the clause for $\kh$ establishes that all the plans in $\plans$ have a probablity of at least $q$ of succeeding. Since all plans in $\plans$ are indistinguishability, this needs to be guaranteed no matter which plan is chosen. However, there are at least two alternatives in this definition: one considering \emph{adaptability}, meaning that there is at least one possible execution of each plan in $\plans$ for which probability $q$ is guaranteed, and considering \emph{non-adaptability} whenever we need this guarantee on each possible execution of every plan in $\plans$. Condition 2 establishes that successfull states are reached with probability of at least $q$.

\begin{remark}
    Notice that an equivalence class $\plans$ can be defined in terms of a Deterministic Finite State Automata (DFA) $\mathcal{A}_{\plans}$. This way, the operation of checking $q$-executability can be perfomed over the product $\model\times\mathcal{A}_{\plans}$. In short, $\plans$ is $q$-executable at $s$ iff for each $\plan\in\plans$, there exists one MDP in  $\model\times\mathcal{A}_{\plans}$ in which the probability of executing $\plan$ is at least $q$. This will be used while designing a model-checking algorithm (connected with algorithms from~\cite{AFSVQ21,AFSVQ23,DF23}).
\end{remark}

\begin{theorem}\label{th:mc-khp-adapt-decidable}
    The model-checking problem for $\PKhunc$ under the semantics from Definition~\ref{def:semantics-adap} is decidable. Moreover, if each $\plans\in\Unc$ is given as input as a deterministic FSA, the problem is in $\PTIME$.
\end{theorem}


\bibliographystyle{plain}
\bibliography{references}

\end{document}