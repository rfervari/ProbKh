\documentclass{article}

\usepackage{settings}

\newtheorem{theorem}{Theorem}
\newtheorem{proposition}{Proposition}%[section]
\newtheorem{corollary}{Corollary}%[section]
\newtheorem{lemma}{Lemma}%[section]
\newtheorem{fact}{Fact}%[section]
\newtheorem{remark}{Remark}%[section]
\newtheorem{claim}{Claim}%[section]

\newtheorem{definition}{Definition}%[section]
\newtheorem{example}{Example}%[section]

%------------------------------------------------------------------------------------------------

\usepackage{cleveref}

% Abbreviated Cref
\Crefname{algorithm}{Alg.}{Algs.}
\Crefname{definition}{Def.}{Defs.}
% \Crefname{equation}{Eq.}{Eqs.}
% \Crefname{fact}{Fact}{Facts}
\Crefname{figure}{Fig.}{Figs.}
\Crefname{proposition}{Prop.}{Props.}
% \Crefname{lemma}{Lemma}{Lemmas}
\Crefname{theorem}{Thm.}{Thms.}
\Crefname{example}{Ex.}{Exs.}
\Crefname{corollary}{Cor.}{Cors.}
% \Crefname{enumi}{Item}{Items}
\Crefname{section}{Sec.}{Secs.}
\Crefname{appendix}{App.}{Apps.}
% \Crefname{table}{Table}{Tables}
% \Crefname{inlineenumi}{Item}{Items}
% % \Crefname{cond-khi}{}{}
% % \Crefname{inline-cond-khi}{}{}
% \Crefname{table}{Tab.}{Tabs.}



%% \usepackage{tikz}
%% \usetikzlibrary{arrows,decorations,shapes,automata,positioning,decorations.pathmorphing,patterns}

\usepackage{tikz, tikzscale, pgfplots}
\usetikzlibrary{arrows,arrows.meta,calc,automata,positioning,decorations.pathreplacing,shapes.geometric,shapes.misc,graphs,backgrounds,shadows.blur,snakes}

\tikzset{align at top/.style={baseline=(current bounding box.north)}}
\tikzstyle{every node}=[font=\scriptsize]

\tikzset{
  every picture/.style = {
    thick,
    ->,
    >=stealth',
    node distance = 1.5em and 3em,
  }
  ,
  cross line/.style = {
    preaction = {
      draw=white,
      -,
      line width=4pt
    }
  }
  , % states
  state/.style = {
    circle,
%    rectangle,
%    rounded corners = 5pt,
    font = \footnotesize,
    draw,
    inner sep = 0pt,
    minimum size = 1.6em
%    minimum width = 1em,
%    minimum height = 1em
  }
  , % nail for distribution
  dot/.style = {
    fill,
    circle,
    inner sep=0mm,
    minimum size=1.25mm,
    line width=0mm
  }
  , % labels of states
  label-state/.style = {
    sloped,
    font = \scriptsize,
    label distance = -2pt
  }
  , % labels of edges
  label-edge/.style = {
    font = \scriptsize,
    label distance = -2pt
  }
}


%------------------------------------------------------------------------------------------------

\newcommand{\powerset}{\mathscr{P}}% requires package mathrsfs
\newcommand{\powersetf}{\powerset_{\mathsf{f}}}

% basic sets
\newcommand{\Prop}{{\rm \sf Prop}\xspace}
\newcommand{\Act}{{\rm \sf Act}\xspace}
\newcommand{\AGT}{{\rm \sf Agt}\xspace}

\newcommand{\FORMS}{{\rm \sf Form}\xspace}

% syntax
\newcommand{\kh}{{\mathsf{Kh}}}
\newcommand{\khi}{{\mathsf{Kh}_i}}
\newcommand{\kc}{{\mathsf{Kc}}}


%\newcommand{\limp}{\rightarrow}
\newcommand{\ra}{\rightarrow}
\newcommand{\lra}{\leftrightarrow}

\newcommand{\liff}{\leftrightarrow}

\newcommand{\A}{{\operatorname{\sf A}}}
\newcommand{\E}{{\operatorname{\sf E}}}

\newcommand{\SAT}{\mathsf{Sat}}

\newcommand{\size}[1]{{|}{#1}{|}}

\newcommand{\Dist}{\mathsf{Dist}}
\newcommand{\support}{\mathsf{Supp}}
\newcommand{\Dirac}{\Delta}
\newcommand{\Cyl}{\mathrm{Cyl}}

\newcommand{\comp}{\circ}


% models
%\newcommand{\modlts}{\mathcal{S}}
% \newcommand{\lmodel}{\mathfrak{L}} % lts model
%\newcommand{\umodel}{\mathfrak{M}} % uncertain model
% \newcommand{\nmodel}{\mathfrak{N}} % normative model
\newcommand{\model}{\mathfrak{M}}
%\newcommand{\moddults}{\mathcal{D}}
% \newcommand{\cults}{\mathcal{C}}
% \newcommand{\clts}{\mathcal{C}_2}
% \newcommand{\canonical}{\model^\Gamma_c}

\newcommand{\LogicLetter}{\mathcal{L}}
\newcommand{\PKh}{{\LogicLetter_{\kh^q}}}
\newcommand{\Khlogic}{{\LogicLetter_{\kh}}}
\newcommand{\Khunc}{{\LogicLetter^{\mathrm{U}}_{\kh}}}
\newcommand{\PKhunc}{{\LogicLetter^{\mathrm{U}}_{\kh^q}}}
\newcommand{\PKhadapt}{{\LogicLetter^{\mathrm{a}}_{\kh^q}}}

\newcommand{\fgetprob}{{\normalfont\textsf{rp}}}

% \newcommand{\R}{\operatorname{R}}
\renewcommand{\S}{\operatorname{S}}
\newcommand{\Unc}{\operatorname{U}}
\newcommand{\V}{\operatorname{V}}


\newcommand{\plan}{\pi}
\newcommand{\plans}{\Pi}
\newcommand{\PLANS}{\ACT^{*}}

\newcommand{\reach}[1]{\xrightarrow{#1}}

\newcommand{\complete}{\bot}
\newcommand{\exec}{\mathit{Exec}}
\newcommand{\cexec}{\mathit{CExec}}
\newcommand{\execf}{\exec_{\mathsf{f}}}
\newcommand{\execw}{\exec_\omega}
\newcommand{\cexecf}{\cexec_{\mathsf{f}}}

\newcommand{\Succ}{\mathit{Succ}}

\newcommand{\Comp}{\mathit{Comp}}

\newcommand{\strat}{\sigma}
\newcommand{\D}[1]{\operatorname{D}_{#1}}
\newcommand{\DS}[1]{\operatorname{D}_{#1}}


% macros for examples

\newcommand{\actionfont}{\mathit}
\newcommand{\lift}{\actionfont{lf}}
\newcommand{\stairs}{\actionfont{st}}
\newcommand{\ramp}{\actionfont{rm}}
\newcommand{\panic}{\actionfont{pn}}
\newcommand{\mobile}{\actionfont{mb}}

\newcommand{\init}{\text{\textcolor{blue!60!black}{\normalfont\textsf{init}}}}
\newcommand{\fin}{\text{\textcolor{green!60!black}{\normalfont\textsf{fin}}}}
\newcommand{\goal}{\text{\textcolor{green!60!black}{\ding{51}}}}
\newcommand{\fail}{\text{\textcolor{red!80!black}{\ding{55}}}}

\newcommand{\modelex}{\ensuremath{\model_{\mathrm{e}}}}

%%

\newcommand{\lts}{\textup{LTS}\xspace}
%\newcommand{\ublts}{\textup{LTS}\xspace}
\newcommand{\nts}{\textup{NLTS}\xspace}

\newcommand{\truthset}[2]{\llbracket #2 \rrbracket^{#1}}

%\newcommand{\cmodel}{\modults^\Gamma}


\newcommand{\iffdef}{\ensuremath{\mbox{\it iff}_{\mbox{\tiny\it  def}}}}

% \newcommand{\sel}{\mathsf{sel}}
% \newcommand{\proj}{\mathsf{pr}}

% notions of executability
\newcommand{\stexec}{\operatorname{SE}}

\newcommand{\last}{\mathrm{last}}
\newcommand{\first}{\mathrm{first}}
\newcommand{\pref}{\mathrm{pref}}
\newcommand{\infim}{\mathrm{inf}}
\newcommand{\Prob}{\mathbb{P}}



% axiom systems
% \newcommand{\axm}[1]{{\rm \textsf{#1}}}
% % \newcommand{\KHaxiom}{\ensuremath{\mathcal{L}^{\lts}_{\kh}}\xspace}
% % \newcommand{\KHiaxiom}{\ensuremath{\mathcal{L}^{\ults}_{\khi}}\xspace}
% % \newcommand{\KCiaxiom}{\ensuremath{\mathcal{L}^{\nts}_{\kci,\obliged,\ability}}\xspace}
% \newcommand{\axset}{\mathcal{DL}}
% \newcommand{\kcaxiom}{\ensuremath{\mathcal{DLK}c}\xspace}


% completeness proof
% \newcommand{\smcs}{\boldsymbol{\Phi}}
% % \newcommand{\restkh}[1]{#1\vert_{\kh}}
% % \newcommand{\restnkh}[1]{#1\vert_{\lnot\kh}}
% % \newcommand{\restkhi}[1]{#1\vert_{\khi}}
% % \newcommand{\restnkhi}[1]{#1\vert_{\lnot\khi}}
% \newcommand{\restkc}[1]{#1|_{{\kc}}}
% \newcommand{\restnkc}[1]{#1|_{{\lnot\kc}}}
% \newcommand{\restkci}[1]{#1|_{{\kci}}}
% \newcommand{\restnkci}[1]{#1|_{{\lnot\kci}}}
% \newcommand{\restn}[1]{#1|_{{\normed}}}
% \newcommand{\restnn}[1]{#1|_{{\lnot\normed}}}
% \newcommand{\rests}[1]{#1|_{{\ability}}}
% \newcommand{\restns}[1]{#1|_{{\lnot\ability}}}
% \newcommand{\resta}[1]{#1|_{{\A}}}
% \newcommand{\restna}[1]{#1|_{{\lnot\A}}}

% \newcommand{\restarbitrary}[1]{#1|_{\arbitrary}}
% \newcommand{\restnarbitrary}[1]{#1|_{\lnot\arbitrary}}

% \newcommand{\resta}[1]{#1\vert_{\A}}
% \newcommand{\restna}[1]{#1\vert_{\lnot\A}}

%------------------------------------------------------------------------------------------------

% utils
\newcommand{\card}[1]{{\mid}{#1}{\mid}}
\newcommand{\tup}[1]{\langle{#1}\rangle}
\newcommand{\cset}[1]{\{{#1}\}}
\newcommand{\csetc}[3]{\{ #1 \in #2 \mid #3 \}}
\newcommand{\csetsc}[2]{\{{#1}\mid {#2}\}}
\newcommand{\setof}[2]{\{{#1}\mid {#2}\}}
\newcommand{\set}[1]{\{{#1}\}}

% \newcommand\SetSymbol[1][]{\nonscript\:#1\vert\allowbreak\nonscript\:\mathopen{}}
% \providecommand\given{} % to make it exist
% \DeclarePairedDelimiterX\Set[1]{\left\{}{\right\}}{\renewcommand\given{\SetSymbol[\delimsize]}#1}


\newcommand{\subformulas}{\mathsf{sf}}
\newcommand{\dmd}{dmd}

% \NewDocumentCommand{\setargs}{>{\SplitArgument{1}{;}}m}
% {\setargsaux#1}
% \NewDocumentCommand{\setargsaux}{mm}
% {\IfNoValueTF{#2}{#1} {#1\nonscript\:\delimsize\vert\allowbreak\nonscript\:\mathopen{}#2}}%
% \def\Set{\set*}%

\newcommand{\ssparagraph}[1]{\smallskip\noindent\textbf{#1}\,}

\newenvironment{smallarray}[1]
 {\null\,\vcenter\bgroup\scriptsize
  \renewcommand{\arraystretch}{0.7}%
  \arraycolsep=.13885em
  \hbox\bgroup$\array{@{}#1@{}}}
 {\endarray$\egroup\egroup\,\null}

%------------------------------------------------------------------------------------------------


% To write notes in the text
\newcommand{\raul}[1]{\todo[color=yellow!55]{{\bf Raul:} #1}\xspace}
\newcommand{\bigraul}[1]{\todo[inline,color=yellow!55]{{\bf Raul:} #1}}

\newcommand{\val}[1]{\todo[color=blue!20]{{\bf Val:} #1}\xspace}
\newcommand{\bigval}[1]{\todo[inline,color=blue!20]{{\bf Val:} #1}}

\newcommand{\pedro}[1]{\todo[color=red!20]{{\bf Pedro:} #1}\xspace}
\newcommand{\bigpedro}[1]{\todo[inline,color=red!20]{{\bf Pedro:} #1}}

% \newcommand{\andres}[1]{\todo[color=cyan!20]{{\bf ARS:} #1}\xspace}
% \newcommand{\bigandres}[1]{\todo[inline,color=cyan!20]{{\bf ARS:} #1}}

\newcommand{\pablo}[1]{\todo[color=green!20]{{\bf Pablo:} #1}\xspace}
\newcommand{\bigpablo}[1]{\todo[inline,color=green!20]{{\bf Pablo:} #1}}

\newcommand{\colornuevo}{teal}
%\newcommand{\lineanueva}[1]{\textcolor{red}{#1}}
\newenvironment{textonuevo}
{\color{\colornuevo}}
{\normalcolor}

\newcommand{\colornota}{Peach}
\newenvironment{notas}
{\smallskip \hlight{NOTES:}\;\color{\colornota}}
{\normalcolor}

\newcommand{\colorincompleto}{red}
\newenvironment{incompleto}
{\color{\colorincompleto}}
{\normalcolor}

% \colorlet{colorhighlight}{Yellow}
% \newcommand{\hlight}[1]{{\setlength{\fboxsep}{2pt}\colorbox{colorhighlight}{#1}}}
% \newcommand{\hlightmath}[1]{{\setlength{\fboxsep}{2pt}\colorbox{colorhighlight}{\ensuremath{#1}}}}

%------------------------------------------------------------------------------------------------

% Complexity classes
\newcommand{\NP}{{\rm\textsf{NP}}\xspace}
\newcommand{\CoNP}{{\rm\textsf{Co-NP}}\xspace}

\newcommand{\PTIME}{{\rm\textsf{PTime}}\xspace}
\newcommand{\PSPACE}{{\rm\textsf{PSpace}}\xspace}
\newcommand{\NPSPACE}{{\rm\textsf{NPSpace}}\xspace}
\newcommand{\EXPTIME}{{\rm\textsf{ExpTime}}\xspace}
\newcommand{\NEXPTIME}{{\rm\textsf{NExpTime}}\xspace}
\newcommand{\LSPACE}{{\rm\textsf{LSpace}}\xspace}
\newcommand{\PH}{{\rm\textsf{PH}}\xspace}

%------------------------------------------------------------------------------------------------

\renewcommand{\iff}{\ensuremath{\mathrel{\text{iff}}}}

\newcommand{\tset}[1]{\llbracket #1 \rrbracket}

% \newcommand{\zerodisplayskips}{%
%   \setlength{\abovedisplayskip}{5pt}%
%   \setlength{\belowdisplayskip}{5pt}%
%   \setlength{\abovedisplayshortskip}{5pt}%
%   \setlength{\belowdisplayshortskip}{5pt}}
% \appto{\normalsize}{\zerodisplayskips}
% \appto{\small}{\zerodisplayskips}
% \appto{\footnotesize}{\zerodisplayskips}

% \DeclareMathOperator{\dom}{dom}
% \DeclareMathOperator{\img}{img}
% \DeclareMathOperator{\depth}{\mathsf{md}}
% \DeclareMathOperator{\sforms}{\mathsf{sf}}
% \DeclareMathOperator{\nnf}{nnf}
% \DeclareMathOperator{\cnf}{cnf}
% \DeclareMathOperator{\C}{\mathrm{\Pi}}
% \DeclareMathOperator{\even}{even}
% \DeclareMathOperator{\odd}{odd}
% \DeclareMathOperator{\zeros}{zeros}

\newcommand{\CPL}{\ensuremath{\mathsf{CPL}}}

% \providecommand{\lxor}{\oplus}%{\veebar}


\begin{document}

\title{How lucky are you to know your way}
\author{Valentin Cassano, Pablo F. Castro, Pedro R. D'Argenio and Raul Fervari}
\date{}


\maketitle

\begin{abstract}
    In this paper we introduce a probabilistic version of knowing how modalities. More precisely, our logics extend existing approaches to model the ability of an agent to achieve a goal, with a certain probability. Models of the logic are extended with probability distributions over the actions that can be executed. In turn, we investigate different variant logics with this new feature. First, we enrich with probabilities the linear plan-based logic of knowing how. Then, we consider indistinguishability classes, and obtain two logics, one that we call with `non-adaptable' plans, and another called with `adaptable' plans. For all these logics we investigate the computational complexity of their model-checking problem, obtaining uncedidability results for the first and the second logic, while for the last one the problem is decidable in polynomial time.
\end{abstract}

\section{Introduction}
\label{sec:intro}

Here we list some important pieces of work, motivating ours:

\begin{itemize}
    \item Knowing how has been investigated is the last years, from different perspectives, especially by combining epistemic operators of knowing that with operators describing abilities~\cite{Mccarthy69,Moore85,Les00,Hoek00,HerzigT06}. This is not considered as a proper reading~\cite{JamrogaA07,Herzig15}
    \item In~\cite{Wang15lori,Wang16,Wang2016} a new perspective on knowing how emerged, in which a new modality is specifically defined with the purpose of capturing this concept.
    \item This raised a family of logics, witnessed by all related work (see e.g.~\cite{LiWang17,Li17,Li17bis,FervariHLW17,LiW21,NaumovT17,NaumovT18,NaumovT19,Naumov2018a}).
    \item A notion of `epistemic indistinguishability' is missing, arguably fixed by~\cite{AFSVQ21,AFSVQ23}.
    \item With this at hand, it was possible to define dynamic epistemic modalities (e.g.~\cite{AFSV22}).
    \item Constraints on plans, like regularity or budget constraints~\cite{DemriF23}.
    \item The latter opens the path to study other constraints, in particular, \emph{knowing how to achieve a goal with a certain probability}.
    \item Relate to other epistemic based logics with probabilities, and with the version of knowing how with uncertainty~\cite{NaumovT19}. Recall the differences, and the case of use that we are able to capture.
    \item Relate to planning with probabilities.
    \item Model-checking with probabilities \cite{BaierAFK18}, related to ATL \cite{BA95,TJ07}, strategy logics \cite{AKMM19}
    \item Recall the different versions of our modality, how we obtain a decidable logic, and why it makes sense.
    \item Connections with reinforcement learning and reasoning about such scenarios.
\end{itemize}

\section{Preliminaries}

\begin{definition}\label{def:plts}
    Let $\PROP$ be a countable set of propositional symbols and let $\AGT$ be a finite set of agents.  
    A \emph{Probabilistic Labeled Transition System (PLTS)} $\model$ is a tuple
    $\tup{\S,\ACT,\dist(S),\ra,\sim,\V}$ such that:
    \begin{itemize}
        \item $\S$ is a countable set of states,
        \item $\ACT$ is a set of action symbols,
        \item $\dist(S)$ is the set of probability distributions over $\S$,
        \item $\ra \subseteq \S \times \ACT \times \dist(\S)$ is a transition relation,
        \item $\sim\subseteq \DS{i} \times \AGT \times \DS{i}$ is an indistinguishability relation between plans for each agent over $\DS{i}\subseteq$, and
        \item $\V: \S \ra 2^\PROP$ is a valuation function.
    \end{itemize}
    We denote by $[\plan]_{\sim_i}:=\set{\plan' \mid \plan \sim_i \plan'}$ and $\Unc(i) := \set{[\plan]_{\sim_i} \mid \plan\in\DS{i}}$. The set $\Unc:=\set{\Unc(i) \mid i\in\AGT}$ is called the  \emph{uncertainty set} of $\model$. For simplicity sake, we sometimes denote $\model=\tup{\S,\ACT,\dist(S),\ra,\Unc,\V}$ to refer to a PLTS, i.e., we will use its uncertainty set instead of the indistinguishability relation.
\end{definition}

\begin{definition}
    \label{def:syntax}
    The set of formulas (a.k.a. the language) of $\PKh$ is defined by the following BNF:
    \[
        \varphi, \psi ::= p \mid \neg \varphi \mid \varphi \vee \psi \mid \kh_i^\rho(\psi,\varphi),
    \]
    where $p\in\PROP$, $i\in\AGT$ and $\rho\in[0,1]$. Other Boolean operators are defined as usual. Formulas of the form $\kh^\rho(\psi,\varphi)$ are read as \emph{``aget $i$ knows how to achieve $\varphi$ given $\psi$, with probability $\rho$''}
\end{definition}

\begin{definition} \label{def:semantics}
    Let $\model = \tup{\S,\ACT,\dist(S),\ra,\Unc,\V}$ be a PLTS and let $s\in\S$, the satisfiability relation $\models$ for $\PKh$ is inductively defined as:
    \[
    \begin{array}{l@{\ \ \ }c@{\ \ \  }l}
    \model, s \models p & \iffdef & p \in \V(s) \\
    \model, s\models \neg\varphi & \iffdef & \model, s \not\models \varphi \\
    \model, s \models \psi\vee\varphi & \iffdef & \model, s \models \psi \mbox{ or }\model, w \models \varphi \\
    \model, s \models \khi(\psi,\varphi) & \iffdef & \text{there is } \plans \in \Unc(i) \;\text{such that:} \\
    & & \ \ \text{\rm (1)} \ \plans \text{ is $\rho$-executable at }  \truthset{\model}{\psi}\; \text{and} \\
    & & \ \ \text{\em (2)} \ \truthset{\model}{\psi} \reach{\plans} \truthset{\model}{\varphi},
    \end{array}
    \]    
    where: $\truthset{\model}{\chi} := \csetsc{s\in\S}{\model,w\models\chi}$. Define: $\model\models\varphi$ iff  $\truthset{\model}{\varphi}=\S$, and $\models\varphi$ iff $\model\models\varphi$, for all PLTS $\model$.
\end{definition}
\section{Knowing How with Linear Plans}
\label{sec:khlinearplans}

\subsection{Basic Definitions}

We start by introducing the most basic notion of knowing how as defined in e.g.~\cite{Wang15lori,Wang16,Wang2016}. Formulas describing the abilities of an agent of achieving a certain goal, are interpreted over Labeled Transition Systems, which indicate what actions are available for execution at each state, and how they transform one state into another.

In order to determine when an agent knows how to achieve a goal, we need to characterize those sequences of actions (or plans) that result appropriate for such a purpose. This is the notion of \emph{strongly executable} plans about, indicating that a plan is ``fail proof''. This notion, discussed already in~\cite{Wang15lori,Wang16,Wang2016} was inspired by conformant planning (see e.g.~\cite{Smith&Weld98,Bonet2010}).

\begin{definition}\label{def:plans}
    Let $\model=\tup{\S,\Act,\ra,\V}$ be an LTS. 
    Elements of $\Act^*$ are called \emph{plans} (with $\epsilon$ the empty plan).  Let $\plan\in\Act^*$, $\size{\plan}$ denotes its length ($\size{\epsilon}:=0$).
    For  $0\leq i \leq \size{\plan}$, the plan $\plan_i$ denotes the initial segment of $\plan$ up to (and including) the $i^{th}$ position (with $\plan_0 := \epsilon$). The action $\plan[i]$ is the one appearing in $\plan$ at the $i^{th}$ position. We define $\reach{\plan}$ as the composition $\reach{\plan[1]} \comp \ldots \comp \reach{\plan[\size{\plan}]}$. 

<<<<<<< HEAD
    We say that a plan $\plan\in\Act^*$ is \emph{strongly executable (SE)} at a state $s\in\S$ if and only if, for all $0\leq i \leq \size{\plan}-1$ and all $t\in\S$ such that $s\reach{\plan_i} t$, there is $v\in\S$ such that $t\reach{\plan[i+1]} v$. The plan $\plan$ is SE at $A\subseteq \S$ if and only if it is SE at every $s\in A$. The notation $A \reach{\plan} G$ (for $A,G\subseteq\S$) indicates that for all $s\in A$,  $s\reach{\plan} t$ implies $t\in G$.
=======
    We say that a plan $\plan\in\Act^*$ is \emph{strongly executable (SE)} at a state $s\in\S$ if and only if, for all $0\leq i \leq \size{\plan}-1$\pedro{deber\'ia ser $1\leq i \leq \size{\plan}-1$} and all $t\in\S$ such that $s\reach{\plan_i} t$, there is $v\in\S$ such that $t\reach{\plan[i+1]} v$. The plan $\plan$ is SE at $T\subseteq \S$ if and only if it is SE at every $s\in T$. The notation $U \reach{\plan} T$ (for $U,T\subseteq\S$) indicates that for all $s\in U$,  $s\reach{\plan} t$ implies $t\in T$.
>>>>>>> refs/remotes/origin/main
\end{definition}

Now we are ready to introduce the language of knowing how.

\begin{definition}
    \label{def:syntax}
    The set of formulas (a.k.a. the language) of $\Khlogic$ is defined by the following BNF:
    \[
        \varphi, \psi ::= p \mid \neg \varphi \mid \varphi \vee \psi \mid \kh(\psi,\varphi),
    \]
    where $p\in\Prop$. Other Boolean operators are defined as usual. Formulas of the form $\kh(\psi,\varphi)$ are read as \emph{``the agent knows how to achieve $\varphi$ given $\psi$''}
\end{definition}

Formulas are interpreted over pointed LTS, i.e., w.r.t. an LTS and a given state. 

\begin{definition} \label{def:semantics-kh}
    Let $\model = \tup{\S,\Act,\ra,\V}$ be an LTS and let $s\in\S$, the satisfiability relation $\models$ for $\Khlogic$ is inductively defined as:
    \[
    \begin{array}{l@{\ \ \ }c@{\ \ \  }l}
    \model, s \models p & \iffdef & p \in \V(s) \\
    \model, s\models \neg\varphi & \iffdef & \model, s \not\models \varphi \\
    \model, s \models \psi\vee\varphi & \iffdef & \model, s \models \psi \mbox{ or }\model, w \models \varphi \\
    \model, s \models \kh(\psi,\varphi) & \iffdef & \text{there is } \plan \in \Act^* \;\text{such that:} \\
    & & \ \ \text{\rm (1)} \ \plan \text{ is SE at }  \truthset{\model}{\psi}\; \text{and} \\
    & & \ \ \text{\em (2)} \ \truthset{\model}{\psi} \reach{\plan} \truthset{\model}{\varphi}, 
    \end{array}
    \]      where: $\truthset{\model}{\chi} := \csetsc{s\in\S}{\model,w\models\chi}$. Define: $\model\models\varphi$ iff  $\truthset{\model}{\varphi}=\S$, and $\models\varphi$ iff $\model\models\varphi$, for all LTS $\model$.
\end{definition}

The model-checking problem for a given logic is defined as follows, where models and formulas are instantiated with those corrresponding to each particular case. 

\begin{description} \itemsep 0cm
    \item[Input:] A model $\model$, a state $s$ in $\model$ and a formula $\varphi$;
    \item[Output:] $\model,s\models\varphi$?
\end{description}

\begin{proposition}[\cite{DemriF23}]
    The model-checking problem for $\Khlogic$ is \PSPACE-complete.
\end{proposition}

\subsection{Indistinguishability Classes in Knowing How}

\begin{definition}[Uncertainty-based \lts]\label{def:ults}
    An \emph{uncertainty-based \lts} (LTSU) for $\Prop$, $\Act$ and $\AGT$ is a tuple     $\model=\tup{\S,\Act,\ra,\sim,\V}$ s.t. $\tup{\S,\Act,\ra,\V}$ is an LTS, and ${\sim}\subseteq \DS{}\times \DS{}$ (where $\DS{}\subseteq\Act^*$) is an equivalence relation over $\DS{}$ (called the indistinguishability relation between plans). 

    By $[\plan]_{\sim}:=\set{\plan' \in \DS{} \mid \plan \sim_i \plan'}$ we denote $\plan$'s equivalence relation with respect to $\DS{}$, then we define the \emph{indistinguishability set of $\model$} as $\Unc := \set{[\plan]_{\sim} \mid \plan\in\DS{}}$. 
        For simplicity sake, we sometimes denote $\model=\tup{\S,\Act,\ra,\Unc,\V}$ to refer to an LTSU, i.e., we will use its uncertainty set instead of the indistinguishability relation.
    \end{definition}
    
    Intuitively, $\DS{} = \bigcup_{\plans \in \Unc} \plans$ is the set of plans that the  agent is aware she has at her disposal, and each $\plans \in \Unc$ is an indistinguishability class. 
    
    Given her uncertainty over $\Act^*$, the abilities of the agent depend not on what a single plan can achieve, but rather on what a set of them can guarantee.
    
    \medskip
    
    \begin{definition}
       Let $\plans \subseteq \Act^*$ and $A\cup G\cup\set{s,t} \subseteq \S$, we write $s\reach{\plans} t$ whenever  $s \reach{\plan} t$ for some $\plan\in\plans$, and $A \reach{\plans} G$ whenever for all $s\in A$, $s\reach{\plans} t$ implies $t\in G$. 
    \end{definition}
    
    \medskip
    
    In what follows, we introduce the notion of strong executability of plans (see, e.g.,~\cite{Wang15lori,AFSVQ23}), a condition which determines that a given plan or a set of them, is appropriate in order to achieve a certain goal.
    
    \medskip
    
    
    \begin{definition}[Strong executability of plans]\label{def:plans-exec}
    Let $\model=\tup{\S,\Act,\ra,\Unc,\V}$ be an LTSU. A \emph{set of plans} $\plans \subseteq \Act^*$ is \emph{strongly executable} at $u \in \S$ if and only if \emph{every} plan $\plan \in \plans$ is \emph{strongly executable} at~$u$.
    % $\stexec(\plans) = \bigcap_{\plan \in \plans} \stexec(\plan)$ is the set of the states in $\W$ where $\plans$ is SE.
    \end{definition}

    \pedro{esta sem\'antica no se condice con la definida en la tesis de Andr\'es}
    \begin{definition} \label{def:semantics-kh-uncertain}
        Let $\model = \tup{\S,\Act,\ra,\Unc,\V}$ be an LTSU and let $s\in\S$, the satisfiability relation $\models$ for $\Khlogic$ is inductively defined as:
        \[
        \begin{array}{l@{\ \ \ }c@{\ \ \  }l}
        \model, s \models \kh(\psi,\varphi) & \iffdef & \text{there is } \plans \in \Unc \;\text{such that:} \\
        & & \ \ \text{\rm (1)} \ \plans \text{ is SE at }  \truthset{\model}{\psi}\; \text{and} \\
        & & \ \ \text{\em (2)} \ \truthset{\model}{\psi} \reach{\plans} \truthset{\model}{\varphi}, 
        \end{array}
        \]      where: $\truthset{\model}{\chi} := \csetsc{s\in\S}{\model,w\models\chi}$. Define: $\model\models\varphi$ iff  $\truthset{\model}{\varphi}=\S$, and $\models\varphi$ iff $\model\models\varphi$, for all LTS $\model$.
    \end{definition}
    

    \begin{proposition}[\cite{AFSVQ21,AFSVQ23}]
        The model-checking problem for $\Khlogic$ over LTSU is in \PTIME.
    \end{proposition}

\section{Knowing How over Indistinguishability classes}
\label{sec:kh:indistinguishability}


\subsection{A First Approach: Non-Adaptable Plans}

Here we introduce some definitions, extending those in e.g.~\cite{AFSVQ21,AFSVQ23}.


\begin{definition}\label{def:plts}
    Let $\PROP$ be a countable set of propositional symbols and let $\AGT$ be a finite set of agents.  
    A \emph{Probabilistic Labeled Transition System with uncertainty (PLTSU)}  is a tuple
    $\model=\tup{\S,\ACT,\ra,\sim,\V}$ s.t. $\tup{\S,\ACT,\ra,\V}$ is a PLTS, and 
    \begin{itemize}
        % \item $\S$ is a countably non-empty set of states,
        % \item $\ACT$ is a countable set of action symbols,
        % \item $\dist(\S)$ is the set of probability distributions over $\S$,
        % \item ${\ra} \subseteq \S \times \ACT \times \dist(\S)$ is a transition relation,
        \item ${\sim}\subseteq \DS{}\times \DS{}$, where $\DS{}\subseteq\ACT^*$,  is an equivalence relation over $\DS{}$, and 
        % \item $\V: \S \ra 2^\PROP$ is a valuation function.
    \end{itemize}
    %Elements of $\ACT^*$ are called \emph{plans}, and 
    Each $\sim$ is called the indistinguishability relation between plan. 
    % We write $\plan\sim_i\plan'$ whenever $(\plan,i,\plan')\in{\sim}$, and we define $\DS{i}:=\set{\plan \mid \text{ there is } \plan' \text{ s.t. } \plan\sim_i\plan'}$. 
    By $[\plan]_{\sim}:=\set{\plan' \in \DS{} \mid \plan \sim_i \plan'}$ we denote $\plan$'s equivalence relation with respect to $\DS{}$, then we define the \emph{indistinguishability set of $\model$} as $\Unc := \set{[\plan]_{\sim} \mid \plan\in\DS{}}$. 
    %The set $\Unc:=\set{\Unc(i) \mid i\in\AGT}$ is called the \emph{uncertainty set} of $\model$. 
    For simplicity sake, we sometimes denote $\model=\tup{\S,\ACT,\ra,\Unc,\V}$ to refer to a PLTS, i.e., we will use its uncertainty set instead of the indistinguishability relation.
\end{definition}

Notice that, as it is defined in~\cite{AFSVQ21,AFSVQ23}, $\Unc$ represents the perception the agent has about the reality. In turn, the relation $\sim$ is not an equivalence relation over $\ACT^*$ but over $\DS{}$, as the latter contains only the plans she considers available or suitable for her purposes, while those in $\ACT^*\setminus\DS{}$ are not considered by the agent, even if they are suitable plans. 

% \begin{definition} \label{def:executability} \raul{TBC}
%     Let  $\model=\tup{\S,\ACT,\dist(\S),\ra,\Unc,\V}$ be a PLTSU, let $\plan\in\ACT^*$ be a plan, and let $q\in[0,1]$ a probability, we need to define two notions:
%     \begin{enumerate}
%         \item We say $\plan$ is \emph{$q$-executable} at $s\in\S$ iff ...  Extend it for set of plans and set of states.
%         \item For $s,t\in\S$, we write $s \reach{\plan}_q t$ iff ... Extend it for set of plans and set of states.
%     \end{enumerate}
% \end{definition}

In general, we will assume that PLTS are complete, as for those we can always complete the execution of a plan to a \emph{sink state}.

% \begin{definition}
%     \label{def:syntax}
%     The set of formulas (a.k.a. the language) of $\PKhunc$ is defined by the following BNF:
%     \[
%         \varphi, \psi ::= p \mid \neg \varphi \mid \varphi \vee \psi \mid \kh_i^q(\psi,\varphi),
%     \]
%     where $p\in\PROP$, $i\in\AGT$ and $q\in[0,1]$. Other Boolean operators are defined as usual. Formulas of the form $\kh_i^q(\psi,\varphi)$ are read as \emph{``agent $i$ knows how to achieve $\varphi$ given $\psi$, with probability $q$''}
% \end{definition}

\begin{definition} \label{def:semantics-non-adap}
    Let $\model = \tup{\S,\ACT,\ra,\Unc,\V}$ be a PLTSU and let $s\in\S$, the satisfiability relation $\models$ for $\PKhunc$ is inductively defined as:
    \[
    \begin{array}{l@{\ \ \ }c@{\ \ \  }l}
    % \model, s \models p & \iffdef & p \in \V(s) \\
    % \model, s\models \neg\varphi & \iffdef & \model, s \not\models \varphi \\
    % \model, s \models \psi\vee\varphi & \iffdef & \model, s \models \psi \mbox{ or }\model, w \models \varphi \\
    \model, s \models \kh^q(\psi,\varphi) & \iffdef & \text{there is } \plans \in \Unc \;\text{such that for all } \plan\in\plans{:} \\
    & & \ \ \text{\rm (1)} \ \plan \text{ is $q$-executable at }  \truthset{\model}{\psi}\; \text{and} \\
    & & \ \ \text{\em (2)} \ \truthset{\model}{\psi} \reach{\plan}_q \truthset{\model}{\varphi}, 
    \end{array}
    \]     \raul{1 and 2 to be merge into one.}
    \noindent where: $\truthset{\model}{\chi} := \csetsc{s\in\S}{\model,w\models\chi}$. Define: $\model\models\varphi$ iff  $\truthset{\model}{\varphi}=\S$, and $\models\varphi$ iff $\model\models\varphi$, for all PLTS $\model$.
\end{definition}

This definition bears a resemblance to the one in Section~\ref{sec:khlinearplans}, except that we ask the conditions hold for \emph{every plan} belonging to a set of indistinguishable plans $\plans$. This way, we obtain a result similar to the one in the previous section.


\subsection{A Second Approach: Adaptable Plans}

\begin{theorem}\label{th:mc-khp-nadapt-undecidable}
    The model-checking problem for $\PKhunc$ under the semantics from Definition~\ref{def:semantics:non-adap} is undecidable.
\end{theorem}

\begin{definition} \label{def:semantics-adap}
    Let $\model = \tup{\S,\ACT,\ra,\Unc,\V}$ be a PLTSU and let $s\in\S$, the satisfiability relation $\models$ for $\PKhunc$ is inductively defined as:
    \[
    \begin{array}{l@{\ \ \ }c@{\ \ \  }l}
    % \model, s \models p & \iffdef & p \in \V(s) \\
    % \model, s\models \neg\varphi & \iffdef & \model, s \not\models \varphi \\
    % \model, s \models \psi\vee\varphi & \iffdef & \model, s \models \psi \mbox{ or }\model, w \models \varphi \\
    \model, s \models \kh^q(\psi,\varphi) & \iffdef & \text{there is } \plans \in \Unc \;\text{such that:} \\
    & & \ \ \text{\rm (1)} \ \plans \text{ is $q$-executable at }  \truthset{\model}{\psi}\; \text{and} \\
    & & \ \ \text{\em (2)} \ \truthset{\model}{\psi} \reach{\plans}_q \truthset{\model}{\varphi}, 
    \end{array}
    \]     \raul{1 and 2 to be merge into one.}
    \noindent where: $\truthset{\model}{\chi} := \csetsc{s\in\S}{\model,w\models\chi}$. Define: $\model\models\varphi$ iff  $\truthset{\model}{\varphi}=\S$, and $\models\varphi$ iff $\model\models\varphi$, for all PLTS $\model$.
\end{definition}

The general idea in condition 1 of the clause for $\kh$ establishes that all the plans in $\plans$ have a probablity of at least $q$ of succeeding. Since all plans in $\plans$ are indistinguishability, this needs to be guaranteed no matter which plan is chosen. However, there are at least two alternatives in this definition: one considering \emph{adaptability}, meaning that there is at least one possible execution of each plan in $\plans$ for which probability $q$ is guaranteed, and considering \emph{non-adaptability} whenever we need this guarantee on each possible execution of every plan in $\plans$. Condition 2 establishes that successfull states are reached with probability of at least $q$.

\begin{remark}
    Notice that an equivalence class $\plans$ can be defined in terms of a Deterministic Finite State Automata (DFA) $\mathcal{A}_{\plans}$. This way, the operation of checking $q$-executability can be perfomed over the product $\model\times\mathcal{A}_{\plans}$. In short, $\plans$ is $q$-executable at $s$ iff for each $\plan\in\plans$, there exists one MDP in  $\model\times\mathcal{A}_{\plans}$ in which the probability of executing $\plan$ is at least $q$. This will be used while designing a model-checking algorithm (connected with algorithms from~\cite{AFSVQ21,AFSVQ23,DF23}).
\end{remark}

\begin{theorem}\label{th:mc-khp-adapt-decidable}
    The model-checking problem for $\PKhunc$ under the semantics from Definition~\ref{def:semantics-adap} is decidable. Moreover, if each $\plans\in\Unc$ is given as input as a deterministic FSA, the problem is in $\PTIME$.
\end{theorem}


\bibliographystyle{plain}
\bibliography{references}

\end{document}