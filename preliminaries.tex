\section{Preliminaries}
\label{sec:preliminaries}

Here we introduce some definitions, extending those in e.g.~\cite{AFSVQ21}.

\begin{definition}\label{def:plts}
    Let $\PROP$ be a countable set of propositional symbols and let $\AGT$ be a finite set of agents.  
    A \emph{Probabilistic Labeled Transition System (PLTS)} $\model$ is a tuple
    $\tup{\S,\ACT,\dist(S),\ra,\sim,\V}$ such that:
    \begin{itemize}
        \item $\S$ is a countable set of states,
        \item $\ACT$ is a set of action symbols,
        \item $\dist(\S)$ is the set of probability distributions over $\S$,
        \item $\ra \subseteq \S \times \ACT \times \dist(\S)$ is a transition relation,
        \item $\sim\subseteq \DS{i} \times \AGT \times \DS{i}$ is an indistinguishability relation between plans for each agent over $\DS{i}\subseteq$, and
        \item $\V: \S \ra 2^\PROP$ is a valuation function.
    \end{itemize}
    We denote by $[\plan]_{\sim_i}:=\set{\plan' \in \DS{i} \mid \plan \sim_i \plan'}$ is $\plan$'s equivalence relation with respect to $\DS{i}$  and define $\Unc(i) := \set{[\plan]_{\sim_i} \mid \plan\in\DS{i}}$. The set $\Unc:=\set{\Unc(i) \mid i\in\AGT}$ is called the  \emph{uncertainty set} of $\model$. For simplicity sake, we sometimes denote $\model=\tup{\S,\ACT,\dist(\S),\ra,\Unc,\V}$ to refer to a PLTS, i.e., we will use its uncertainty set instead of the indistinguishability relation.
\end{definition}

\begin{definition}
    \label{def:syntax}
    The set of formulas (a.k.a. the language) of $\PKh$ is defined by the following BNF:
    \[
        \varphi, \psi ::= p \mid \neg \varphi \mid \varphi \vee \psi \mid \kh_i^\rho(\psi,\varphi),
    \]
    where $p\in\PROP$, $i\in\AGT$ and $\rho\in[0,1]$. Other Boolean operators are defined as usual. Formulas of the form $\kh_i^\rho(\psi,\varphi)$ are read as \emph{``agent $i$ knows how to achieve $\varphi$ given $\psi$, with probability $\rho$''}
\end{definition}

\begin{definition} \label{def:semantics}
    Let $\model = \tup{\S,\ACT,\dist(\S),\ra,\Unc,\V}$ be a PLTS and let $s\in\S$, the satisfiability relation $\models$ for $\PKh$ is inductively defined as:
    \[
    \begin{array}{l@{\ \ \ }c@{\ \ \  }l}
    \model, s \models p & \iffdef & p \in \V(s) \\
    \model, s\models \neg\varphi & \iffdef & \model, s \not\models \varphi \\
    \model, s \models \psi\vee\varphi & \iffdef & \model, s \models \psi \mbox{ or }\model, w \models \varphi \\
    \model, s \models \kh_i^\rho(\psi,\varphi) & \iffdef & \text{there is } \plans \in \Unc(i) \;\text{such that:} \\
    & & \ \ \text{\rm (1)} \ \plans \text{ is $\rho$-executable at }  \truthset{\model}{\psi}\; \text{and} \\
    & & \ \ \text{\em (2)} \ \truthset{\model}{\psi} \reach{\plans}_\rho \truthset{\model}{\varphi}, 
    \end{array}
    \]     \raul{$\rho$-executable and $ \reach{\plans}_\rho$ to be defined}
    where: $\truthset{\model}{\chi} := \csetsc{s\in\S}{\model,w\models\chi}$. Define: $\model\models\varphi$ iff  $\truthset{\model}{\varphi}=\S$, and $\models\varphi$ iff $\model\models\varphi$, for all PLTS $\model$.
\end{definition}