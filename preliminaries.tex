\section{Preliminaries}
\label{sec:preliminaries}


% Languages and traces

\paragraph{Words.} Let $\Act$ be a finite set of \emph{actions}, also called
\emph{alphabet} and let $\plan\in\Act^*$ be a word with alphabet
$\Act$.  We use $\size{\plan}$ to denote the length of $\plan$ (with
$\size{\epsilon}=0$) and $\plan[i]$, for $0<i\leq\size{\plan}$,
to denote the $i$-th element of $\plan$.
%
For $0\leq i\leq\size{\plan}$, $\plan[..i]$ denotes the $i$-th prefix
of $\plan$, i.e., the initial segment of $\plan$ up to (and including)
the $i$-th position (with $\plan[..0] = \epsilon$).
%
We say that $\plan$ is a \emph{prefix} of $\plan'$, denoted
$\plan\leq\plan'$, if $\plan=\plan'[..i]$ for some $0\leq
i\leq\size{\plan}$.  For a set $\plans\subseteq\Act^*$, let
$\pref(\plans)=\{\rho'\in\Act^*\mid{\exists{\rho\in\plans}\colon{\rho'\leq\rho}}\}$
be the set of all prefixes of $\plans$.  %In particular, 
We write $\pref(\plan)$ for $\pref(\{\plan\})$.
%
Besides, $\plan\plan'$ denotes the concatenation of $\plan$ followed by
$\plan'$.
%% Besides, $\plan\plan'$ denotes the concatenation of $\plan$ and
%% $\plan'$ and if $\plans\subseteq\Act^*$,
%% $\plan\plans=\{\plan\plan'\mid{\plan'\in\plans}\}$.

% Probabilities

\paragraph{(Probabilistic) Labeled Transition Systems.} 
A (discrete) \emph{probability distribution} $\mu$ over a denumerable
set $S$ is a function $\mu: S \rightarrow [0, 1] $ such that
$\mu(S) = \sum_{s \in S} \mu(s) = 1$.  Let $\Dist(S)$ denote the set
of all probability distributions on $S$, and let $\Dirac_s \in \Dist(S)$
denote the Dirac distribution for $s \in S$, i.e., $\Dirac_s(s) =1$
and $\Dirac_s(s') = 0$ for all $s' \in S$ such that $s'\neq s$.
%% The
%% \textit{support} set of $\mu$ is defined by
%% $\support{\mu} = \{s \mid {\mu(s) > 0}\}$.


% PLTS / LTS

Let $\Prop$ be a countable set of propositional symbols.
%
A \emph{probabilistic labeled transition System
(PLTS)}~\cite{Segala95} is a tuple $\model=\tup{\S,\Act,\ra,\V}$ where
$\S$ is a finite set of \emph{states}, $\Act$ is a finite set of
\emph{actions}, ${\ra}\subseteq {\S \times \Act \times \Dist(\S)}$ is
the \emph{(probabilistic) transition relation}, and
$\V:\S\to\powerset(\Prop)$ is the \emph{valuation function}.
%
We denote $s\reach{a}\mu$ whenever $(s,a,\mu)\in{\ra}$ and let
$T(s)=\{(a,\mu)\mid {s\reach{a}\mu}\}$ be the set of all transitions
enabled in state $s$.
%
For the case in which, for all $(s,a,\mu)\in{\ra}$, $\mu$ is a Dirac
distribution $\Dirac_{s'}$ for some $s'\in\S$, we say that $\model$ is
a \emph{labeled transition system (LTS)} and denote $s\reach{a}s'$
instead of $s\reach{a}\Dirac_{s'}$.
\pedro{hay que ver si queda esta definicion de LTS}

% Executions, strategies and probability on executions

An \emph{execution} of $\model$ is a finite or infinite alternating
sequence of states and actions $s_0\, a_1\, s_1\, a_2\, s_2\ldots$.
$\execf= \S\times(\Act\times\S)^*$ denotes the set of all
\emph{finite executions} and $\execw= \S\times(\Act\times\S)^\omega$
denotes the set of all \emph{infinite executions}.
%
We introduce the symbol $\complete$ to indicate that a finite
execution has been intentionally ended and let $\cexecf=
\S\times(\Act\times\S)^*\times\{\complete\}$ denote the set of all
\emph{complete finite executions}.  Infinite executions are also
considered complete and hence $\cexec=\cexecf\cup\execw$ is the
set of all \emph{complete executions}.

For $\rho = s_0\, a_1\, s_1\ldots s_{n-1}\, a_n\, s_n \in \execf$ and
$0\leq k\leq n$, let $\size{\rho}=n$, $\first(\rho)=s_0$ and
$\last(\rho)=s_n$.
%
Let also $\rho[..k] = s_0\, a_1\, s_1\ldots s_{k-1}\, a_k\, s_k$ be
the $k$-th prefix of $\rho$.
%
Similarly, we can define $\rho[..k]$ for $\rho\in\cexec$.
In particular, notice that $\rho[..0]=s_0$.
%
We say that $\rho \in \execf\cup\cexecf$ is a \emph{prefix} of $\rho' \in \cexec$,
denoted by $\rho\leq\rho'$, if $\rho=\rho'[..\size{\rho}]$ or $\rho=\rho'$
(this last case is needed if $\rho\in\cexecf$).
%
(The overloading of notation with respect to words should be harmless
and easily understood by context.)

A \emph{strategy} for a PLTS $\model$ is a function $\strat:
\execf \to \Dist((\Act\times\Dist(S))\cup\{\complete\})$ that assigns
a discrete probability distribution to each finite (non-complete)
execution $\rho\in\execf$ such that $\strat(\rho)(a,\mu) > 0$ only
if $\last(\rho)\reach{a}\mu$.  Thus, a strategy can choose with some
probability a valid transition after $\rho$ or to intentionally
terminate (in case $\strat(\rho)(\complete)>0$).

Let $\Cyl(\rho)=\{\rho'\in\cexec\mid{\rho\leq\rho'}\}$ be the
\emph{cylynder set} induced by the finite execution
$\rho\in\execf\cup\cexecf$.  Notice that we only consider cylinders of
complete executions and in particular $\Cyl(\rho)=\{\rho\}$ whenever
$\rho\in\cexecf$.
%
A strategy $\strat$ and a state $s\in\S$ define a probability
measure $\Prob^\strat_s$ on the Borel sigma algebra generated by the
set of all cylinder sets as follows. \raul{es necesario poner 'on the Borel..' o alcanza con definir la medida de probabilidad?}
%
For $\rho = s_0\, a_1\, s_1\ldots s_{n-1}\, a_n\, s_n \in \execf$,
%
\begin{align}
  &\Prob^\strat_s(\Cyl(\rho)) = {}\label{eq:def:prob:i}\\
  &\quad \Dirac_s(s_0)\cdot\prod\limits_{i=1}^n\sum\limits_{(a_i,\mu)\in T(s_{i-1})}\strat(\rho[..(i{-}1)])(a_i,\mu)\cdot\mu(s_i)\notag
\end{align}
%
and for
$\rho = s_0\, a_1\, s_1\ldots s_{n-1}\, a_n\, s_n\, \complete \in \cexecf$,
%
\begin{align}
  &\Prob^\strat_s(\Cyl(\rho)) = {}\label{eq:def:prob:ii}\\
  &\quad \Dirac_s(s_0)\cdot\bigg(\prod\limits_{i=1}^n\sum\limits_{(a_i,\mu)\in T(s_{i-1})}\strat(\rho[..(i{-}1)])(a_i,\mu)\cdot\mu(s_i)\bigg)\notag\\
  &\phantom{\quad \Dirac_s(s_0)} \cdot \ \strat(\rho)(\complete).\notag
\end{align}
%
Charathedeory's extension theorem guarantees that $\Prob^\strat_s$ is
uniquely defined in the sigma algebra~\cite{Segala95}.


% PFA


\paragraph{Probabilistic Finite Automata.} A PLTS is deterministic if for all $s\in\S$, $a\in\Act$, and
$\mu_1,\mu_2\in\Dist(\S)$, if $s\reach{a}\mu_1$ and $s\reach{a}\mu_2$
then $\mu_1=\mu_2$.
%
A \emph{probabilistic finite automata} (PFA)~\cite{Rabin63,Paz71} is a
deterministic PLTS $\pfa=\tup{\S,\Act,\ra,\V}$ on the set
$\Prop=\{\init,\fin\}$, i.e., $\V:\S\to\powerset(\{\init,\fin\})$,
such that there is a single state $\sini\in\S$, called
\emph{initial}, with $\init\in\V(\sini)$.
%
$F=\{{s\in\S}\mid{\fin\in\V(s)}\}$ is the set of \emph{final} or
\emph{accepting} states.

Given an execution
$\rho = s_0\,a_1\,s_1\ldots s_{n-1}\,a_n\,s_n \in \execf$,
let $\bar{\rho} = a_1\, a_2\ldots a_n$.  In particular
$\bar{\rho} = \epsilon$ whenever $n=0$.
%
For a word $\plan\in\Act^*$, define the strategy $\strat_\plan$ such
that for all $\rho\in\execf$ with $\bar{\rho}\leq\plan$,
%
\begin{enumerate*}[(i)]
\item%
  $\strat_\plan(\rho)(a,\mu)=1$ iff $\bar{\rho}a\leq\plan$ (in which
  case $\mu$ is unique), and
\item%
  $\strat_\plan(\rho)(\complete)=1$ iff either $\bar{\rho}=\plan$ or
  $\bar{\rho}\{a\mid{\last(\rho)\reach{a}\mu}\}\cap\pref(\plan) = \emptyset$.
\end{enumerate*}
%
Since the PFA is deterministic, for any $s\in\S$, $\strat_\plan$
defines the same probability measure $\Prob_s^{\strat_\plan}$
regardless of its definition for all $\rho$ such that
$\bar{\rho}\not\leq\plan$.) 
%
The word $\plan$ is accepted by the PFA $\pfa$ with probability
$q\in[0,1]$ if $\Prob_{\sini}^{\strat_\plan}(\Succ(\plan,F))\geq q$, where
$\Succ(\plan,F)=\{{\rho\complete\in\cexecf}\mid{\bar{\rho}=\plan \text{ and } \last(\rho)\in F}\}$.
%
Let $L(\pfa,q)$ be the set of all such words.

\begin{proposition}{\cite{Paz71,MadaniHC99}}\label{prop:empty:PFA:undecidable}
  The emptiness problem in PFA (i.e., checking if
  $L(\pfa,q)=\emptyset$) is undecidable.
\end{proposition}


